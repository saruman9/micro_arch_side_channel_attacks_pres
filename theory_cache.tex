\againframe<3>{core_elements}

\begin{frame}{\insertsubsection}


  \begin{figure}[h]
    \begin{tikzpicture}[
      align = center,
      ->,
      > = Stealth,
      thick,
      core/.style = {
        rectangle,
        draw,
        text centered,
        minimum width = 1cm,
        minimum height = 1cm,
      },
      cache1/.style = {
        rectangle split,
        rectangle split horizontal,
        rectangle split parts = 2,
        rectangle split part fill = {
          OliveGreen, Blue
        },
        draw,
        text = White,
      },
      cache2/.style = {
        rectangle,
        draw,
        fill = Sepia,
        text = White,
      },
      cache3/.style = {
        rectangle,
        draw,
        fill = ForestGreen,
        text = White,
        minimum width = 9cm,
      },
      block/.style = {
        rectangle,
        draw,
        fill = Maroon,
        text = White,
      },
      ]

      \node[
      core,
      ] (core_0) {
        Core 0
      };


      \node[
      core,
      right = 2cm of core_0,
      ] (core_1) {
        Core 1
      };

      \node[
      core,
      right = 2cm of core_1,
      ] (core_2) {
        Core 2
      };

      \node[
      core,
      right = 2cm of core_2,
      ] (core_3) {
        Core 3
      };

      \node[
      cache1,
      below = of core_0.center,
      ] (c1_0) {
        32KB,\\L1-i
        \nodepart{two}
        32KB,\\L1-d
      };

      \node[
      cache1,
      below = of core_1.center,
      ] (c1_1) {
        32KB,\\L1-i
        \nodepart{two}
        32KB,\\L1-d
      };

      \node[
      cache1,
      below = of core_2.center,
      ] (c1_2) {
        32KB,\\L1-i
        \nodepart{two}
        32KB,\\L1-d
      };

      \node[
      cache1,
      below = of core_3.center,
      ] (c1_3) {
        32KB,\\L1-i
        \nodepart{two}
        32KB,\\L1-d
      };

      \node[
      cache2,
      below = .5 of c1_0,
      ] (c2_0) {
        256KB,\\L2 (NINE)
      };

      \node[
      cache2,
      below = .5 of c1_1,
      ] (c2_1) {
        256KB,\\L2 (NINE)
      };

      \node[
      cache2,
      below = .5 of c1_2,
      ] (c2_2) {
        256KB,\\L2 (NINE)
      };

      \node[
      cache2,
      below = .5 of c1_3,
      ] (c2_3) {
        256KB,\\L2 (NINE)
      };

      \node[
      cache3,
      below = .5 of $(c2_1.south)!.5!(c2_2.south)$,
      ] (c3) {
        8MB,\\
        shared L3 (inclusive L1, L2)
      };

      \node[
      block,
      minimum width = 2cm,
      right = of c3,
      ] (mc) {MC};

      \node[
      block,
      minimum width = 2cm,
      left = of c3,
      ] (ht) {HT};

      \node[
      block,
      below = .5 of mc,
      ] (dram) {DRAM};

      \node[
      block,
      below = .5 of ht,
      ] (chips) {
        Другие\\
        чипы
      };

      \draw[<->] (core_0) -- (c1_0);
      \draw[<->] (core_1) -- (c1_1);
      \draw[<->] (core_2) -- (c1_2);
      \draw[<->] (core_3) -- (c1_3);

      \draw[<->] (c1_0) -- (c2_0);
      \draw[<->] (c1_1) -- (c2_1);
      \draw[<->] (c1_2) -- (c2_2);
      \draw[<->] (c1_3) -- (c2_3);

      \draw[<->] (c2_0) -- (c3);
      \draw[<->] (c2_1) -- (c3);
      \draw[<->] (c2_2) -- (c3);
      \draw[<->] (c2_3) -- (c3);

      \draw[<->] (c3) -- (ht);
      \draw[<->] (c3) -- (mc);

      \draw[<->] (ht) -- (chips);
      \draw[<->] (mc) -- (dram);

    \end{tikzpicture}
    \caption{Архитектура процессора относительно
      кэшей}\label{fig:proc_cache_arch}
  \end{figure}

  \note{

    Современные процессоры имеют целую \textbf{иерархию кэшей} с различными
    размерами и скоростью обращения. Некоторые кэши приватные и работают в
    контексте \textbf{только одного процессора}, \textbf{некоторые общие}, их
    могут читать и писать все процессоры.

    Существует несколько \textbf{правил включения (инклюзивности) кэша} один в
    другой: правило \textbf{инклюзивности}, \textbf{эксклюзивности},
    \textbf{NINE}.

    HT --- HyperTransport; MC --- Memory Controller.

  }

\end{frame}

\begin{frame}{\insertsubsection}


  \begin{figure}[h]

    \newcommand*\virtcolor{ForestGreen, ForestGreen, ForestGreen, ForestGreen, ForestGreen, ForestGreen}

    \only<2-3>{

      \renewcommand*\virtcolor{ForestGreen, ForestGreen, BrickRed, ForestGreen, ForestGreen, ForestGreen}

    }

    \begin{tikzpicture}[
      align = center,
      ->,
      > = Stealth,
      thick,
      double = ForestGreen,
      block/.style = {
        draw,
        text = White,
        minimum width = 2cm,
        rectangle split,
        rectangle split parts = 6,
        rectangle split part fill = {
          ForestGreen, ForestGreen, ForestGreen, ForestGreen, ForestGreen, ForestGreen
        },
      },
      cache/.style = {
        draw,
        text = Black,
        text centered,
        minimum width = 2cm,
        rectangle split,
        rectangle split parts = 2,
      },
      ]

      \node[block,
      rectangle split part fill = {\virtcolor},
      label = above:Виртуальная память,
      ] (virt) {
        0xf200
        \nodepart{two}
        0xf100
        \nodepart{three}
        0xf000
        \nodepart{four}
        $\cdots$
        \nodepart{five}
        0x2000
        \nodepart{six}
        0x1000
        \nodepart{seven}
      };

      \node[cache,
      right = of virt,
      label = above:Адрес,
      ] (cache_addr) {
        \only<1-3>{0xf200}%
        \only<4>{0xf000}%
        \nodepart{two}
        0x2000
        \nodepart{three}
      };

      \node[cache,
      right = 0 of cache_addr,
      label = above:Содержимое,
      fill = NavyBlue,
      ] (cache_content) {
        \only<1-3>{\color{White}Kernel secret 0}%
        \only<4>{\color{White}Kernel secret 1}%
        \nodepart{two}
        \color{White}User secret
        \nodepart{three}
      };

      \node[
      rectangle,
      fit = (cache_addr) (cache_content),
      label = {below:Кэш (L1, L2, LLC)},
      inner sep = 0,
      ] {};

      \node[block,
      right = of cache_content,
      label = above:Физическая память\\(DRAM),
      ] (phys) {
        0x1000
        \nodepart{two}
        0x0900
        \nodepart{three}
        0x0800
        \nodepart{four}
        0x0700
        \nodepart{five}
        0x0600
        \nodepart{six}
        0x0500
        \nodepart{seven}
      };

      \draw<1>[ForestGreen] (virt.one east)  -- ++(.5, 0) -- node[right] {hit} ++(0, -1) |- (cache_addr.one west);
      \draw<1>[ForestGreen] (virt.five east) -- ++(.5, 0) -- node[below right] {hit} ++(0, .4) |- (cache_addr.two west);

      \draw<1-3> (phys.three west) -- (cache_content.one east);
      \draw (phys.four west) -- (cache_content.two east);

      \draw<2>[BrickRed] (virt.three east) -- node[above] {miss} (cache_addr.west);

      \draw<3>[BrickRed] (virt.three east) -- ++(.5, 0) -- ++(0, -3) -| (phys.south);

      \draw<4> (virt.three east) -- ++(.5, 0) -- ++(0, -3) -| (phys.south);
      \draw<4>[BrickRed] (phys.six west) -- ++(-.3, 0) |- (cache_content.one east);

    \end{tikzpicture}
    \caption{Пример взаимодействия с
      кэшем}\label{fig:cache_manipulation_example}
  \end{figure}

  \note<1>{

    В общем случае, все доступы к памяти происходят через кеш. Если доступ к
    памяти происходит через кеш, то это называется \textbf{попаданием кэша}
    (\textit{cache hit}).

  }

  \note<2> {

    В противном случае происходит \textbf{промах кэша} (\textit{cache miss}).

  }

  \note<3> {

    Данные берутся из медленной памяти.

  }

  \note<4> {

    В кэш записываются новые данные.

  }

\end{frame}

\subsubsection{Кэш с прямым отображением}
\begin{frame}{\insertsubsubsection}


  \begin{figure}[h]
    \begin{tikzpicture}[
      align = center,
      ->,
      > = Stealth,
      thick,
      ampersand replacement = \&,
      mem/.style = {
        draw,
        minimum width = 2cm,
        rectangle split,
        rectangle split parts = 3,
        rectangle split horizontal,
        rectangle split every empty part = {},
        rectangle split empty part width = width("n бит"),
      },
      cache/.style = {
        matrix of nodes,
        nodes in empty cells,
        nodes = {
          minimum width = 3cm,
          minimum height = .8cm,
          draw,
          align = center,
          anchor = center,
        },
        right delimiter = \},
      },
      ]

      \node[cache,
      label = above:Кэш,
      label = {
        [label distance = 10]right:$2^n$\\линий\\кэша
      },
      ] (cache) {
        Тег \& Данные \\
        \& \\
        \& \\
        \& \\
      };

      \node[
      below = of cache-3-2,
      ] (b_bits) {
        Смещение в данных на $2^b$ байт
      };

      \node[mem,
      left = of cache-1-1.west,
      label = above:Адрес памяти,
      ] (addr) {
        \nodepart{two}
        \textit{n} бит
        \nodepart{three}
        \textit{b} бит
        \nodepart{four}
      };

      \node[
      draw,
      dashed,
      below = of addr.one south,
      rectangle,
      minimum width = 1cm,
      minimum height = 1cm,
      ] (f) {
        f
      };

      \node[
      circle,
      draw,
      below = of cache-4-1,
      ] (hit) {
        hit?
      };

      \node[below = .5 of hit] (final) {};

      \draw (addr.one south) -> (f);
      \draw (f) |- node[below right] {Тег} (hit);
      \draw (hit) -> (final);
      \draw (cache-3-1.center) -> (hit);
      \draw (addr.two south) |- node[below right] {Индекс кэша} (cache-3-1.west);
      \draw (b_bits) -> (cache-3-2.center);


    \end{tikzpicture}
    % \caption{Схема работы кэша с прямым
    %   отображением}\label{fig:directly_mapped_cache}
  \end{figure}

  \note{

    Кэш состоит из $2^n$ \textbf{линий кэша}, каждая линия содержит \textbf{тег}
    и $2^b$ байт \textbf{ассоциированных данных}. Тег вычисляется из
    соответствующего адреса памяти, который добавляется в эту кэш линию.
    \textbf{Тег используется} в дальнейшем для того, чтобы определять
    \textbf{присутствие} того или иного адреса в линии кэша. Последние
    \textit{b} бит адреса \textbf{используются} в качестве \textbf{смещения для
      данных} в линии кэша. Современные процессоры имеют длину линии кэша в
    \textit{64 байта}, т. е. \texttt{b = 6}. \textbf{Средние} \textit{n} бит
    адреса памяти \textbf{используются в качестве \textit{индекса кэша}},
    который говорит о номере линии кэша, в котором содержатся данные.

    Размер кэша определяет, как много бит будет использовано, т. е. как много
    индексов будет использовано. \textbf{Адреса с теми же \textit{n} битами}
    являются \textbf{конгруэнтными}, так как они отображают те же линии кэша.

    Главная \textbf{проблема} такого вида кэша --- это то, что кэш может
    \textbf{хранить единственную} линию кэша из всех конгруэнтных.
    Следовательно, если процессору требуется работать с двумя или более
    конгруэнтными линиями кэша, то такого рода кэш будет совершать множество
    промахов.

  }

\end{frame}

\subsubsection{Полностью ассоциативный кэш}
\begin{frame}{\insertsubsubsection}

  \newcommand*\nodeonecolor{White}
  \newcommand*\nodetwocolor{White}
  \newcommand*\nodethreecolor{White}
  \newcommand*\arrowonecolor{Black}
  \newcommand*\arrowtwocolor{Black}
  \newcommand*\arrowthreecolor{Black}

  \only<2>{
    \renewcommand*\nodeonecolor{Red}
    \renewcommand*\arrowonecolor{Red}
  }
  \only<3>{
    \renewcommand*\nodetwocolor{Red}
    \renewcommand*\arrowtwocolor{Red}
  }
  \only<4>{
    \renewcommand*\nodethreecolor{ForestGreen}
    \renewcommand*\arrowthreecolor{ForestGreen}
  }

  \begin{figure}[h]
    \begin{tikzpicture}[
      align = center,
      ->,
      > = Stealth,
      thick,
      ampersand replacement = \&,
      mem/.style = {
        draw,
        minimum width = 2cm,
        rectangle split,
        rectangle split parts = 2,
        rectangle split horizontal,
        rectangle split every empty part = {},
        rectangle split empty part width = width("bbbbbbbbbbbbbbb"),
      },
      cache/.style = {
        matrix of nodes,
        nodes in empty cells,
        nodes = {
          minimum width = 3cm,
          minimum height = .8cm,
          draw,
          align = center,
          anchor = center,
        },
      },
      ]

      \node[cache,
      label = above:Кэш,
      ] (cache) {
        Тег \& Данные \\
        \& \\
        \& \\
        \& \\
      };

      \node[mem,
      left = of cache-1-1.west,
      label = above:Адрес памяти,
      ] (addr) {
        \nodepart{two}
        \textit{b} бит
        \nodepart{three}
      };

      \node[
      draw,
      dashed,
      below = of addr.one south,
      rectangle,
      minimum width = 1cm,
      minimum height = 1cm,
      ] (f) {
        f
      };

      \node[
      fill = \nodeonecolor,
      text = Black,
      circle,
      draw,
      below = 2cm of cache-2-1,
      ] (hit_1) {
        \only<1>{hit?}
        \only<2->{miss}
      };

      \node[
      fill = \nodetwocolor,
      text = Black,
      circle,
      draw,
      below right = 0.1 of hit_1,
      ] (hit_2) {
        \only<1-2>{hit?}
        \only<3->{miss}
      };

      \node[
      fill = \nodethreecolor,
      text = Black,
      circle,
      draw,
      below right = 0.1 of hit_2,
      ] (hit_3) {
        \only<1-3>{hit?}
        \only<4->{hit!}
      };

      \node<4>[below = of cache-4-2] (data) {
        Данные
      };

      \draw (addr.one south) -> (f);
      \draw (f) |- node[below right] {Тег} (hit_1);
      \draw (f) |- (hit_2);
      \draw (f) |- (hit_3);
      \draw[\arrowonecolor] (cache-2-1.center) -> (hit_1);
      \draw[\arrowtwocolor] ([xshift = 0.8cm]cache-3-1.center) -> (hit_2);
      \draw[\arrowthreecolor] ([xshift = 1cm]cache-4-1.center) -> (hit_3);
      \draw<4>[\arrowthreecolor] (cache-4-2.center) -> (data);


    \end{tikzpicture}
    % \caption{Схема работы полностью ассоциативного
    %   кэша}\label{fig:fully_associative_cache}
  \end{figure}

  \note<1>{

    Проблема конгруэнтности \textbf{решается в полностью ассоциативном кэше}.

    Такой вид кэша не содержит индексов и каких-либо линий кэша. Вместо этого он
    хранит множество \textbf{путей кэша}, которые в свою очередь содержат
    данные. \textbf{Тег} теперь \textbf{используется для определения
      существования адреса в кэше} и какой именно путь кэша содержит
    ассоциированные данные.

    \textbf{Пример на следующих слайдах!}

  }

  \note<2->{

    Такие кэши становятся более \textbf{дорогими с увеличением путей}. Поэтому
    они обычно содержат небольшое количество путей, например, в современных
    процессорах используются \textbf{буферы ассоциативной трансляции
      (translation-lookaside buffers TLB)} с 64 путями.

  }

\end{frame}

\subsubsection{Наборно--ассоциативный кэш}
\begin{frame}{\insertsubsubsection}

  \begin{figure}[h]
    \begin{tikzpicture}[
      align = center,
      ->,
      > = Stealth,
      thick,
      ampersand replacement = \&,
      mem/.style = {
        draw,
        minimum width = 2cm,
        rectangle split,
        rectangle split parts = 3,
        rectangle split horizontal,
        rectangle split every empty part = {},
        rectangle split empty part width = width("n бит"),
      },
      cache/.style = {
        matrix of nodes,
        nodes in empty cells,
        nodes = {
          font = \scriptsize,
          minimum width = 3cm,
          minimum height = .6cm,
          draw,
          align = center,
          anchor = center,
        },
        text = Black,
        right delimiter = \},
      },
      ]

      \node[cache,
      label = above:Кэш,
      label = {
        [label distance = 10]right:$2^n$\\наборов\\кэша
      },
      ] (cache) {
        \node[fill = black!30]{Тег 1-го пути}; \& \node[fill = black!30]{Данные 1-го пути}; \\
        \node[fill = black!10]{Тег 2-го пути}; \& \node[fill = black!10]{Данные 2-го пути}; \\
        \node[fill = black!30](tag_1){}; \& \node[fill = black!30](data_1){}; \\
        \node[fill = black!10](tag_2){}; \& \node[fill = black!10](data_2){}; \\
        \node[fill = black!30]{}; \& \node[fill = black!30]{}; \\
        \node[fill = black!10]{}; \& \node[fill = black!10]{}; \\
        \node[fill = black!30]{}; \& \node[fill = black!30]{}; \\
      };

      \node[mem,
      left = of cache-1-1.west,
      label = above:Адрес памяти,
      ] (addr) {
        \nodepart{two}
        \textit{n} бит
        \nodepart{three}
        \textit{b} бит
        \nodepart{four}
      };

      \node[
      draw,
      dashed,
      below = of addr.one south,
      rectangle,
      minimum width = 1cm,
      minimum height = 1cm,
      ] (f) {
        f
      };

      \node[
      circle,
      draw,
      below = of cache-4-1,
      ] (hit_1) {
        hit?
      };

      \node[
      circle,
      draw,
      below right = 0.1 of hit_1,
      ] (hit_2) {
        hit?
      };

      \draw (addr.one south) -> (f);
      \draw (f) |- node[below right] {Тег} (hit_1);
      \draw (f) |- node[below right] {}    (hit_2);
      \draw (addr.two south) |- node[below right] {Индекс\\ набора кэша} (cache.west);

      \draw (tag_1.center) -> (hit_1);
      \draw ([xshift = 0.8cm]tag_2.center) -> (hit_2);

    \end{tikzpicture}
    % \caption{Схема работы кэша с ассоциативным набором
    %   (сетом)}\label{fig:set_associative_cache}
  \end{figure}

  \note{

    Компромиссом между этими двумя видами кэша оказывается \textbf{кеш с
      наборами}, а не с линиями кэша. Данные кэши широко используются в
    современных процессорах, где их называют \textbf{\textit{m}-путейные (или
      \textit{m}-входовые) кэши с ассоциативным набором}. Рисунок отображает
    абстрактную модель 2-путейного кэша данного вида.

    Кэш делится на $2^n$ набора. \textbf{Индекс набора} в кэше определяется
    средними \textbf{n} битами адреса. Каждый набор имеет \textbf{m путей} для
    возможности хранения местоположения \textbf{m конгруэнтных адресов}. Наборы
    кэша могут быть также представлены в виде крошечного полностью
    ассоциативного кэша с \textit{m} путями для набора конгруэнтных адресов.
    Поэтому \textbf{тег} снова используется для определения какой путь кэша
    \textbf{содержит определённый адрес}.

  }

\end{frame}

\subsubsection{Правила вымещения из кэша}
\begin{frame}{\insertsubsubsection}

  \begin{itemize}
    \item \texttt{FIFO}
    \item \texttt{LIFO}
    \item least recently used, \texttt{LRU}
    \item time aware least recently used, \texttt{TLRU}
    \item most recently used, \texttt{MRU}
    \item pseudo-LRU, \texttt{PLRU}
    \item random replacement, \texttt{RR}
    \item segment LRU, \texttt{SLRU}
    \item least frequently used, \texttt{LFU}
    \item least frequent recently used, \texttt{LFRU}
    \item LFU with dynamic aging, \texttt{LFUDA}
    \item low inter--reference recency set, \texttt{LIRS}
    \item adaptive replacement cache, \texttt{ARC}
    \item clock with adaptive replacement, \texttt{CAR}
    \item multi queue, \texttt{MQ}
    \item и другие.
  \end{itemize}

  \note{

    \textbf{Количество путей или линий в кэше ограничено}, а
    \textbf{конгруэнтных адресов}, которые требуется хранить, ---
    \textbf{достаточно много}, требуется производить замены данных в кэше на
    новые, полученные из главной памяти.

    Производители процессоров хранят детали этих правил в \textbf{секрете}, так
    как данные правила очень сильно влияют на скорость работы процессора в
    целом.

    Самое широкое распространение получили правила вытеснения
    \textbf{«вытеснение давно неиспользуемых» (least-recently used, LRU)}.

    Процессоры \textbf{ARM} обычно используют правила \textbf{случайного
      вымещения}, так как такие правила просто реализовать на аппаратных
    средствах, и в ходе своей работы они потребляют мало энергии, а также
    показывают себя высокопроизводительными.

  }

\end{frame}

\subsubsection{Режимы адресации}
\begin{frame}{\insertsubsubsection. VIVT}

  \begin{figure}[h]
    \begin{tikzpicture}[
      align = center,
      ->,
      > = Stealth,
      thick,
      ampersand replacement = \&,
      mem/.style = {
        draw,
        minimum width = 2cm,
        rectangle split,
        rectangle split parts = 3,
        rectangle split horizontal,
        rectangle split every empty part = {},
        rectangle split empty part width = width("n бит"),
      },
      cache/.style = {
        matrix of nodes,
        nodes in empty cells,
        nodes = {
          font = \scriptsize,
          minimum width = 3cm,
          minimum height = .6cm,
          draw,
          align = center,
          anchor = center,
        },
        right delimiter = \},
        text = Black,
      },
      ]

      \node[cache,
      label = above:Кэш,
      label = {
        [label distance = 10]right:$2^n$\\наборов\\кэша
      },
      ] (cache) {
        \node[fill = black!30]{Тег 1-го пути}; \& \node[fill = black!30]{Данные 1-го пути}; \\
        \node[fill = black!10]{Тег 2-го пути}; \& \node[fill = black!10]{Данные 2-го пути}; \\
        \node[fill = black!30](tag_1){}; \& \node[fill = black!30](data_1){}; \\
        \node[fill = black!10](tag_2){}; \& \node[fill = black!10](data_2){}; \\
        \node[fill = black!30]{}; \& \node[fill = black!30]{}; \\
        \node[fill = black!10]{}; \& \node[fill = black!10]{}; \\
        \node[fill = black!30]{}; \& \node[fill = black!30]{}; \\
      };

      \node[mem,
      left = of cache-1-1.west,
      label = above:Виртуальный адрес памяти,
      ] (addr) {
        \nodepart{two}
        \textit{n} бит
        \nodepart{three}
        \textit{b} бит
        \nodepart{four}
      };

      \node[
      draw,
      dashed,
      below = of addr.one south,
      rectangle,
      minimum width = 1cm,
      minimum height = 1cm,
      ] (f) {
        f
      };

      \node[
      circle,
      draw,
      below = of cache-4-1,
      ] (hit_1) {
        hit?
      };

      \draw (addr.one south) -> (f);
      \draw (addr.two south) |- node[below right] {Индекс\\ набора кэша} (cache.west);

      \draw (f) |- node[below right] {Тег} (hit_1);

      \draw (tag_1.center) -> (hit_1);

    \end{tikzpicture}
    \caption{Виртуальная индексация виртуальное
      тагетирование}\label{fig:vivt_cache}
  \end{figure}

  \note{

    Кэши могут использовать как \textbf{виртуальные адреса}, так и
    \textbf{физические для вычисления индекса кэша и тега}. На практике
    используется три способа вычисления данных.

    \begin{itemize}
    \item Виртуальный тег не уникален при переключении контекста ---
      \textbf{данные не могут быть разделяемыми}
    \item Трансляция адреса не происходит --- \textbf{быстрая скорость работы}
    \end{itemize}

    \textbf{Виртуальная индексация виртуальное тагетирование (virtually-indexed
      virtually-tagged VIVT)}. Используется для маленьких данных, с которыми
    производятся быстрые операции, в \textbf{ARM процессорах} используются в
    качестве \textbf{кэша инструкций}.

  }
\end{frame}

\begin{frame}{\insertsubsubsection. PIPT}

  \begin{figure}[h]
    \begin{tikzpicture}[
      align = center,
      ->,
      > = Stealth,
      thick,
      ampersand replacement = \&,
      mem/.style = {
        draw,
        minimum width = 2cm,
        rectangle split,
        rectangle split parts = 3,
        rectangle split horizontal,
        rectangle split every empty part = {},
        rectangle split empty part width = width("n бит"),
      },
      cache/.style = {
        matrix of nodes,
        nodes in empty cells,
        nodes = {
          font = \scriptsize,
          minimum width = 3cm,
          minimum height = .6cm,
          draw,
          align = center,
          anchor = center,
        },
        right delimiter = \},
        text = Black,
      },
      ]

      \node[cache,
      label = above:Кэш,
      label = {
        [label distance = 10]right:$2^n$\\наборов\\кэша
      },
      ] (cache) {
        \node[fill = black!30]{Тег 1-го пути}; \& \node[fill = black!30]{Данные 1-го пути}; \\
        \node[fill = black!10]{Тег 2-го пути}; \& \node[fill = black!10]{Данные 2-го пути}; \\
        \node[fill = black!30](tag_1){}; \& \node[fill = black!30](data_1){}; \\
        \node[fill = black!10](tag_2){}; \& \node[fill = black!10](data_2){}; \\
        \node[fill = black!30]{}; \& \node[fill = black!30]{}; \\
        \node[fill = black!10]{}; \& \node[fill = black!10]{}; \\
        \node[fill = black!30](point){}; \& \node[fill = black!30]{}; \\
      };

      \node[mem,
      left = of cache-1-1.west,
      label = above:Виртуальный адрес памяти,
      ] (addr) {
        \nodepart{two}
        \textit{n} бит
        \nodepart{three}
        \textit{b} бит
        \nodepart{four}
      };

      \node[
      draw,
      dashed,
      below = .5 of addr,
      xshift = -.6cm,
      ] (tlb) {
        TLB
      };

      \node[mem,
      below = 1.6 of addr,
      ] (addr_phys) {
        \nodepart{two}
        \textit{n} бит
        \nodepart{three}
        \textit{b} бит
        \nodepart{four}
      };

      \node[
      draw,
      dashed,
      below = .5 of addr_phys.one south,
      rectangle,
      minimum width = 1cm,
      minimum height = 1cm,
      ] (f) {
        f
      };

      \node[
      circle,
      draw,
      below = of cache-4-1,
      ] (hit_1) {
        hit?
      };

      \draw (addr_phys.one south) -> (f);
      \draw (addr_phys.two south) |- node[below right] {Индекс\\ набора кэша} (point);

      \draw (addr.205) -> (tlb);
      \draw (tlb) -> (addr_phys.155);
      \draw (addr.three south) -> (addr_phys.three north);

      \draw (f) |- node[below right] {Тег} (hit_1);

      \draw (tag_1.center) -> (hit_1);

    \end{tikzpicture}
    \caption{Физическая индексация физическое
      тагетирование}\label{fig:pipt_cache}
  \end{figure}

  \note{

    \textbf{Физическая индексация физическое тагетирование (physically-indexed
      physically-tagged PIPT)}. Используется \textbf{тег и индекс из физического
      адреса}.

    \begin{itemize}
    \item Тег будет уникальным даже при смене контекста --- \textbf{разделяемая
        память будет реально разделяемой}
    \item Смена контекста --- \textbf{большие задержки}
    \end{itemize}

    Используется для кэшей данных и инструкций, задержка по большей части
    уменьшается посредством использования кэшей в системе трансляции адресов
    (TLB).

  }

\end{frame}

\begin{frame}{\insertsubsubsection. VIPT}

  \begin{figure}[h]
    \begin{tikzpicture}[
      align = center,
      ->,
      > = Stealth,
      thick,
      ampersand replacement = \&,
      mem/.style = {
        draw,
        minimum width = 2cm,
        rectangle split,
        rectangle split parts = 3,
        rectangle split horizontal,
        rectangle split every empty part = {},
        rectangle split empty part width = width("n бит"),
      },
      cache/.style = {
        matrix of nodes,
        nodes in empty cells,
        nodes = {
          font = \scriptsize,
          minimum width = 3cm,
          minimum height = .6cm,
          draw,
          align = center,
          anchor = center,
        },
        right delimiter = \},
        text = Black,
      },
      ]

      \node[cache,
      label = above:Кэш,
      label = {
        [label distance = 10]right:$2^n$\\наборов\\кэша
      },
      ] (cache) {
        \node[fill = black!30]{Тег 1-го пути}; \& \node[fill = black!30]{Данные 1-го пути}; \\
        \node[fill = black!10]{Тег 2-го пути}; \& \node[fill = black!10]{Данные 2-го пути}; \\
        \node[fill = black!30](tag_1){}; \& \node[fill = black!30](data_1){}; \\
        \node[fill = black!10](tag_2){}; \& \node[fill = black!10](data_2){}; \\
        \node[fill = black!30]{}; \& \node[fill = black!30]{}; \\
        \node[fill = black!10]{}; \& \node[fill = black!10]{}; \\
        \node[fill = black!30]{}; \& \node[fill = black!30]{}; \\
      };

      \node[mem,
      left = of cache-1-1.west,
      label = above:Виртуальный адрес памяти,
      ] (addr) {
        \nodepart{two}
        \textit{n} бит
        \nodepart{three}
        \textit{b} бит
        \nodepart{four}
      };

      \node[
      draw,
      dashed,
      below = .5 of addr,
      xshift = -.6cm,
      ] (tlb) {
        TLB
      };

      \node[mem,
      below = 1.6 of addr,
      ] (addr_phys) {
        \nodepart{two}
        \textit{n} бит
        \nodepart{three}
        \textit{b} бит
        \nodepart{four}
      };

      \node[
      draw,
      dashed,
      below = .5 of addr_phys.one south,
      rectangle,
      minimum width = 1cm,
      minimum height = 1cm,
      ] (f) {
        f
      };

      \node[
      circle,
      draw,
      below = of cache-4-1,
      ] (hit_1) {
        hit?
      };

      \draw (addr_phys.one south) -> (f);
      \draw (addr.two south) |- node[below right] {Индекс\\ набора кэша} (tag_1);

      \draw (addr.205) -> (tlb);
      \draw (tlb) -> (addr_phys.155);
      \draw (addr.three east) -- ++(0.7, 0) -- ++(0, -1) |- (addr_phys.three east);

      \draw (f) |- node[below right] {Тег} (hit_1);

      \draw (tag_1.center) -> (hit_1);

    \end{tikzpicture}
    \caption{Виртуальная индексация физическое
      тагетирование}\label{fig:vipt_cache}
  \end{figure}

  \note{

    \textbf{Виртуальная индексация физическое тагетирование (virtually-indexed
      physically-tagged VIPT).}

    Позволяет использовать \textbf{тег из физического адреса}, при этом
    небольшая задержка, так как для поиска в первую очередь и чаще всего
    требуется определить номер набора, который задан виртуальным адресом.

    \begin{itemize}
    \item уникальный тег --- \textbf{возможность применять разделяемые данные}
    \item всё происходит быстро, потому что \textbf{трансляция} адреса
      происходит \textbf{параллельно} поиску \textbf{индекса кэша}
    \end{itemize}

  }

\end{frame}
