\begin{frame}{\insertsubsection}

  CVE-2017-5754: спекулятивное чтение недоступных данных

  \begin{figure}[h]
    \includegraphics[height = .7\textheight]{meltdown_logo}
    \caption{Meltdown}
  \end{figure}

  \note{


  }
\end{frame}

\subsubsection{Необходимые условия}
\begin{frame}{\insertsubsubsection}
  \begin{itemize}
  \item Конвейерное выполнение кода не по порядку
  \item Трансляция адресов памяти с привилегированными данными
  \item На некоторых процессорах данные также должны лежать в кэше данных L1
  \item Точные счётчики времени
  \item и др.
  \end{itemize}

  \note{

    \begin{itemize}
    \item На самом деле не все архитектуры используют выполнение не по порядку в
      том виде, которое требуется для реализации meltdown.
    \item Должна существовать возможность обращения к привилегированным данным
      по виртуальному адресу памяти.
    \item Всё зависит от того, как происходит работа с кэшем.
    \item Для реализации атаки на кэш.
    \end{itemize}

  }
\end{frame}

% \subsubsection{Выполнение не по порядку}
% \begin{frame}[fragile]{\insertsubsubsection}

%           \begin{minted}[escapeinside = ||]{c}
%             |\tikzmark{firsts}|int width = 10, height = 5;|\tikzmark{firste}|

%             |\tikzmark{seconds}|float diagonal = sqrt(width * width + height * height);|\tikzmark{seconde}|

%             |\tikzmark{thirds}|int area = width * height;|\tikzmark{thirde}|

%             |\tikzmark{fourths}|printf("Area %d x %d = %d\n", width, height, area);|\tikzmark{fourthe}|
%           \end{minted}

%   \begin{tikzpicture}[
%     remember picture,
%     align = center,
%     ->,
%     > = Stealth,
%     thick,
%     ]

%     \node[
%     overlay,
%     above right = and -1cm of pic cs:seconde,
%     ] (paral) {
%       Параллелизация
%     };

%     \draw[overlay] (pic cs:firsts) -- ++(-1, 0) |- node[below] {Зависимости}  (pic cs:fourths);
%     \draw[overlay] (pic cs:thirds) to[bend right] (pic cs:fourths);

%     \draw[overlay] (paral.north) to[bend right] (pic cs:firste);
%     \draw[overlay] (paral) to[bend left] (pic cs:seconde);
%     \draw[overlay] (paral.290) |- (pic cs:thirde);

%   \end{tikzpicture}

%   \note{

%     В разделе теории было уже рассказано, что инструкции могут выполняться не по
%     порядку, а некоторые даже спекулятивно, если у них нет зависимостей.

%   }
% \end{frame}

% \subsubsection{Чтение недоступной памяти}
% \begin{frame}{\insertsubsubsection}

%   \begin{itemize}
%   \item Если программа читает память,то
%     \begin{enumerate}
%     \item проверяются права
%     \item память считывается
%     \end{enumerate}
%   \item Если программа пытается читать недоступную память, то
%     \begin{enumerate}
%     \item происходит ошибка
%     \item выполнение останавливается
%     \end{enumerate}
%   \end{itemize}

%   Но что будет, если проверка прав и чтение будут выполняться не по порядку?

%   \note{


%   }
% \end{frame}

% \begin{frame}[fragile]{\insertsubsubsection}

%   \begin{minted}{c}
%     *(volatile char*) 0; // ошибка чтения, выполнение прерывается
%     temp = array[84 * 4096]; // выполнение вне очереди?
%   \end{minted}

%   \only<2>{С помощью атаки на кэш \texttt{Flush + Reload} выясняется, что
%     обращение к 84 странице памяти состоялось!}

%   \note<1>{

%     \begin{enumerate}
%     \item Попытка чтения недоступной памяти
%     \item Проверка прав на чтение
%     \item Параллельно: спекулятивное выполнение последующей операции
%     \item Параллельно: запись данных в кэш
%     \end{enumerate}

%   }

% \end{frame}

% \subsubsection{Эксплуатация}
% \begin{frame}[fragile]{\insertsubsubsection}

%   \begin{minted}{c}
%     unsigned char value = *(unsigned char *)ptr;
%     unsigned long index = (((value >> bit) & 1) * 0x100) + 0x200;
%     maccess(&data[index]);
%   \end{minted}

%   \note{

%     \textbf{ptr} --- указатель на интересующую нас память.

%     \textbf{data} --- контролируемый атакующим массив.

%   }

% \end{frame}

% \begin{frame}{\insertsubsubsection}

%   \begin{figure}[h]
%     \begin{tikzpicture}[
%       align = center,
%       ->,
%       > = Stealth,
%       thick,
%       ampersand replacement = \&,
%       data/.style = {
%         matrix of nodes,
%         nodes in empty cells,
%         nodes = {
%           minimum width = 3cm,
%           minimum height = .8cm,
%           draw,
%           align = center,
%           anchor = center,
%           fill = ForestGreen,
%         },
%         text = White,
%       },
%       block/.style = {
%         rectangle,
%         draw,
%         fill = ForestGreen,
%         text = White,
%         text centered,
%       },
%       ]

%       \node<1->[
%       block,
%       font = \ttfamily,
%       ] (secret) {
%         char value = *SECRET\_KERNEL\_PTR;
%       };

%       \node<2->[
%       block,
%       below = of secret,
%       ] (mask) {
%         маска для чтения нужного бита
%       };

%       \node<3->[
%       block,
%       below = of mask,
%       ] (offset) {
%         вычисление смещения в \texttt{data}\\(к которому есть доступ)
%       };

%       \node<4->[
%       right = of offset,
%       data,
%       label = { above:\texttt{char data[];}},
%       ] (data) {
%         0x000\&a\\
%         0x100\&b\\
%         0x200\&c\\
%         0x300\&d\\
%       };

%       \draw<2-> (secret) -- (mask);
%       \draw<3-> (mask) -- (offset);
%       \draw<4-> (offset) -- (data-3-1);
%       \draw<4-> (offset) -- (data-4-1);

%     \end{tikzpicture}
%     \only<1>{\caption{\texttt{unsigned char value = \color{NavyBlue}*(unsigned char *)ptr;}}}
%     \only<2>{\caption{\texttt{unsigned long index = ((\color{NavyBlue}(value >{}> bit) \& 1\color{Black}) * 0x100) + 0x200;}}}
%     \only<3>{\caption{\texttt{unsigned long index = (((value >{}> bit) \& 1)\color{NavyBlue} * 0x100) + 0x200;}}}
%     \only<4>{\caption{\texttt{maccess(\&data[index]);}}}

%   \end{figure}

%   \note{


%   }

% \end{frame}

% \begin{frame}{\insertsubsubsection}

%   \begin{figure}[h]
%     \begin{tikzpicture}[
%       align = center,
%       ->,
%       > = Stealth,
%       thick,
%       ampersand replacement = \&,
%       data/.style = {
%         matrix of nodes,
%         nodes in empty cells,
%         nodes = {
%           minimum width = 3cm,
%           minimum height = .8cm,
%           draw,
%           align = center,
%           anchor = center,
%           fill = ForestGreen,
%         },
%         text = White,
%       },
%       block/.style = {
%         rectangle,
%         draw,
%         fill = ForestGreen,
%         text = White,
%         text centered,
%       },
%       ]

%       \node[
%       data,
%       label = {above:\texttt{char data[];}},
%       ] (data) {
%         0x000\&a\\
%         0x100\&b\\
%         0x200\&c\\
%         0x300\&d\\
%       };

%       \node<1>[
%       right = of data-3-2,
%       label = above:Кэш,
%       block,
%       rectangle split,
%       rectangle split parts = 2,
%       rectangle split horizontal,
%       ] (cache) {
%         0x200
%         \nodepart{two}
%         c
%       };

%       \node<2>[
%       right = of data-4-2,
%       label = above:Кэш,
%       block,
%       rectangle split,
%       rectangle split parts = 2,
%       rectangle split horizontal,
%       ] (cache) {
%         0x300
%         \nodepart{two}
%         d
%       };

%       \draw<1> (data-3-2) -- (cache);
%       \draw<2> (data-4-2) -- (cache);

%     \end{tikzpicture}
%     \caption{При обращении к массиву по определённому индексу данные попадают в
%       кэш}

%   \end{figure}

%   \note{


%   }

% \end{frame}

% \subsubsection{ASM}
% \begin{frame}[fragile]{\insertsubsubsection}

%   \begin{minted}{nasm}
%     LDR X1, [X2]       ; X2 - указатель на данные, которых нет в кэше,
%                        ; также в TLB не должно быть данного адреса
%     CBZ X1, over       ; переход, который в итоге будет совершён,
%                        ; но инструкции ниже всё равно исполнятся
%     LDR X3, [X4]       ; X4 - указатель на данные в пространстве памяти ядра
%     LSL X3, X3, #imm   ; получение нужного бита данных
%     AND X3, X3, #0xFC0 ; выравнивание с размером страницы памяти
%     LDR X5, [X6,X3]    ; X6 - адрес массива атакующего
%     over
%   \end{minted}

%   \note{

%     Читаем недоступные пользователю данные из памяти, лучше, если адрес не будет
%     храниться в TLB, также данных не будет в кэше.

%   }

% \end{frame}

\subsubsection{Предотвращение}
\begin{frame}{\insertsubsubsection}

  Изоляция адресного пространства ядра --- kernel page-table isolation, KPTI
  (KEISER --- Kernel Address Isolation to have Side-channels Efficiently
  Removed)

  \note{

    При смене контекста происходит проседание производительности ввиду того, что
    полностью сбрасывается TLB, изменяется CR3 регистр.

  }

\end{frame}