\begin{frame}{\insertsubsection}

  \begin{figure}[h]
    \begin{tikzpicture}[
      align = center,
      ->,
      > = Stealth,
      thick,
      ampersand replacement = \&,
      block/.style = {
        draw,
        fill = ForestGreen,
        text = White,
      },
      cache/.style = {
        rectangle split,
        rectangle split parts = 5,
        draw,
        minimum width = 2cm,
      },
      ]

      \node[
      cache,
      label = above:Cache 0,
      ] (cache_0) {
      };

      \node[
      right = of cache_0,
      draw,
      minimum width = 1cm,
      minimum height = 1cm,
      ] (cpu_0) {
        CPU\\0
      };

      \node[
      cache,
      label = above:Cache 3,
      above = of cache_0,
      ] (cache_3) {
      };

      \node[
      right = of cache_3,
      draw,
      minimum width = 1cm,
      minimum height = 1cm,
      ] (cpu_3) {
        CPU\\3
      };

      \node[
      cache,
      label = above:Cache 1,
      left = of cache_0,
      ] (cache_1) {
      };

      \node[
      left = of cache_1,
      draw,
      minimum width = 1cm,
      minimum height = 1cm,
      ] (cpu_1) {
        CPU\\1
      };

      \node[
      cache,
      label = above:Cache 2,
      above = of cache_1,
      ] (cache_2) {
      };

      \node[
      left = of cache_2,
      draw,
      minimum width = 1cm,
      minimum height = 1cm,
      ] (cpu_2) {
        CPU\\2
      };

      \draw (cpu_0) -- (cache_0);
      \draw (cpu_1) -- (cache_1);
      \draw (cpu_2) -- (cache_2);
      \draw (cpu_3) -- (cache_3);
      \draw[ForestGreen] (cache_0.three west) -- ++(-0.5, 0) |- (cache_3.four west);
      \draw[BrickRed] (cache_1.five east) -- ++(0.3, 0) |- (cache_3.one west);
      \draw[Green] (cache_2.one east) -- ++(0.1, 0) |- (cache_3.three west);
      \draw[Blue] (cache_1.two west) -- ++(-0.5, 0) |- (cache_2.five west);

    \end{tikzpicture}
    \caption{Cross-core covert channels}\label{covert_attack_example}
  \end{figure}

  \note{

    Современные облачные системы часто имеют \textbf{несколько процессоров},
    установленных в материнскую плату с \textbf{мультисокетом}. Кэшы процессоров
    содержатся в когерентном состоянии с помощью \textbf{межпроцессорных
      протоколов, обеспечивающих когерентность}. Однако, это обеспечивает эффект
    \textbf{разделяемой кэш линии}.

    Кеш-атаки позволяют реализовать \textbf{высокопроизводительные межъядерные и
      межпроцессорные скрытые кеш-каналы} на современных смартфонах, используя
    \texttt{Flush+Reload}, \texttt{Evict+Reload} или \texttt{Flush+Flush}.
    Скрытый канал позволяет двум \textbf{непривилегированным приложениям}
    взаимодействовать друг с другом \textbf{без использования} каких-либо
    системных \textbf{механизмов передачи данных}. Благодаря этому можно
    вырваться из песочницы и обойти систему «ограниченных разрешений». В
    частности, на Android злоумышленник может использовать одно приложение,
    которое имеет \textbf{доступ к личным контактам} владельца устройства, для
    \textbf{отправки данных по скрытому каналу} другому приложению, имеющему
    доступ к \textbf{интернет}.

  }

\end{frame}

\begin{frame}{\insertsubsection}
  \begin{itemize}
  \item Cache-based covert channels (shared libraries)
  \item Row miss attack (DRAM)
  \item Thermal covert channels
  \item Radio covert channels
  \end{itemize}

  \note{

    Wu в 2012 и в 2014 годах обнаружил \textbf{разницу во времени при доступе к
      памяти}, возникающую из-за \textbf{задержки в шине памяти}, что позволяет
    наладить скрытый канал передачи данных между \textbf{близко расположенными
      виртуальными машинами}. В облаке Amazon EC2 был налажен канал скоростью
    13.5 КБ/c при 0.75 \% ошибок. В 2016 Inci также обнаружил задержку в шине
    памяти, позволяющую наладить канал в облаках Microsoft Azure.

    В 2017 году была представлена первая в своём роде реализация
    \textbf{скрытого канала, работающего по протоколу SSH}, с относительно
    высокой пропускной способностью (45 Кбит/с); эта реализация обеспечивает
    отказоустойчивые коммуникации между двумя виртуальными машинами даже в
    условиях экстремальной зашумлённости кеша.
    
    
  }
\end{frame}

% \subsubsection{Пример работы}
% \begin{frame}{\insertsubsubsection}

%   \begin{figure}[h]
%     \newcommand*\colorshared{White, White, White,
%       White, White}

%     \only<2>{
%       \renewcommand*\colorshared{White, White,
%         ForestGreen, White, White}
%     }
%     \only<3>{
%       \renewcommand*\colorshared{White, White,
%         ForestGreen, White, White}
%     }
%     \only<4>{
%       \renewcommand*\colorshared{White, ForestGreen,
%         White, White, White}
%     }
%     \only<5>{
%       \renewcommand*\colorshared{White, ForestGreen,
%         White, White, White}
%     }
%     \only<6>{
%       \renewcommand*\colorshared{White, White,
%         ForestGreen, White, White}
%     }
%     \only<7>{
%       \renewcommand*\colorshared{White, ForestGreen,
%         White, White, White}
%     }
%     \only<8>{
%       \renewcommand*\colorshared{White, White,
%         ForestGreen, White, White}
%     }

%     \begin{tikzpicture}[
%       align = center,
%       ->,
%       > = Stealth,
%       thick,
%       ampersand replacement = \&,
%       block/.style = {
%         draw,
%         fill = ForestGreen,
%         text = White,
%       },
%       cache/.style = {
%         rectangle split,
%         rectangle split parts = 5,
%         rectangle split part fill = \colorshared,
%         draw,
%         minimum width = 2cm,
%         text = Black,
%       },
%       ]

%       \node[
%       cache,
%       label = above:Разделяемая\\библиотека,
%       ] (shared) {
%         0x7f8517d35000
%         \nodepart{two}
%         0x7f8517d36000
%         \nodepart{three}
%         0x7f8517d37000
%         \nodepart{four}
%         0x7f8517d38000
%         \nodepart{five}
%         0x7f8517d39000
%         \nodepart{six}
%       };

%       \node[
%       text = Black,
%       draw,
%       left = 2cm of shared,
%       ] (p0) {
%         Процесс 0
%       };

%       \node[
%       above = of p0,
%       text = Black,
%       ] (p0_t) {
%         \only<2>{\color{ForestGreen}1}
%         \only<3>{1\color{ForestGreen}1}
%         \only<4>{11\color{ForestGreen}0}
%         \only<5>{110\color{ForestGreen}0}
%         \only<6>{1100\color{ForestGreen}1}
%         \only<7>{11001\color{ForestGreen}0}
%         \only<8>{110010\color{ForestGreen}1}
%       };

%       \node[
%       text = Black,
%       draw,
%       right = 2cm of shared,
%       ] (p1) {
%         Процесс 1
%       };

%       \node[
%       above = of p1,
%       text = Black,
%       ] (p1_t) {
%         \only<2>{\color{ForestGreen}1}
%         \only<3>{1\color{ForestGreen}1}
%         \only<4>{11\color{ForestGreen}0}
%         \only<5>{110\color{ForestGreen}0}
%         \only<6>{1100\color{ForestGreen}1}
%         \only<7>{11001\color{ForestGreen}0}
%         \only<8>{110010\color{ForestGreen}1}
%       };

%       \draw<1-3,6,8> (p0) -- node[above] {Запись} (shared);
%       \draw (p1) -- node[above] {Чтение} (shared);
%       \draw<4-5,7> (p0) -- node[above] {Запись} (shared.two west);

%     \end{tikzpicture}
%     \caption{Скрытые каналы между приложениями на базе
%       кэшей}
%   \end{figure}

%   \note<1>{

%     Основная идея скрытого канала заключается в том, что \textbf{отправитель и
%       получатель согласовывают набор адресов памяти} какой-нибудь
%     \textbf{разделяемой библиотеки}. Они используют для передачи информации
%     загрузку \textbf{ячейки в кеш или ее выгрузку оттуда}. Например, если
%     такая-то ячейка находится в кеше, то это \textbf{единичка}, а если нет, то
%     \textbf{нолик}. В нескольких работах представлена пакетная реализация
%     передачи данных с возможностью повторного запроса недоставленных пакетов;
%     здесь используется «бит отправки» и «бит подтверждения», которые реализованы
%     по такому же принципу, а также контрольная сумма введена (а-ля \textbf{TCP
%       по скрытому каналу}).

%   }

%   \note<8>{

%     Wu в 2012 и в 2014 годах обнаружил \textbf{разницу во времени при доступе к
%       памяти}, возникающую из-за \textbf{задержки в шине памяти}, что позволяет
%     наладить скрытый канал передачи данных между \textbf{близко расположенными
%       виртуальными машинами}. В облаке Amazon EC2 был налажен канал скоростью
%     13.5 КБ/c при 0.75 \% ошибок. В 2016 Inci также обнаружил задержку в шине
%     памяти, позволяющую наладить канал в облаках Microsoft Azure.

%     В 2017 году была представлена первая в своём роде реализация
%     \textbf{скрытого канала, работающего по протоколу SSH}, с относительно
%     высокой пропускной способностью (45 Кбит/с); эта реализация обеспечивает
%     отказоустойчивые коммуникации между двумя виртуальными машинами даже в
%     условиях экстремальной зашумлённости кеша.

%   }

% \end{frame}

% \subsubsection{DRAM (row miss атака)}
% \begin{frame}{\insertsubsubsection}

%   \begin{figure}[h]

%     \newcommand*\dramcolor{Green, Green, NavyBlue,
%       NavyBlue, White, White, White}

%     \newcommand*\bufcolor{White}

%     \only<2>{
%       \renewcommand*\bufcolor{Green}
%     }

%     \only<3>{
%       \renewcommand*\bufcolor{Green}
%     }

%     \only<4>{
%       \renewcommand*\bufcolor{NavyBlue}
%     }

%     \only<5>{
%       \renewcommand*\bufcolor{Green}
%     }

%     \begin{tikzpicture}[
%       align = center,
%       ->,
%       > = Stealth,
%       thick,
%       dram/.style = {
%         rectangle,
%         rectangle split,
%         rectangle split parts = 7,
%         rectangle split part fill = \dramcolor,
%         minimum width = 3cm,
%         draw,
%         text = Black,
%         text centered,
%       },
%       block/.style = {
%         rectangle,
%         draw,
%         fill = ForestGreen,
%         text = White,
%         text centered,
%       },
%       ]

%       \node[
%       dram,
%       label = above:DRAM банк,
%       ] (dram) {
%         0123456789
%         \nodepart{two}
%         1234567890
%         \nodepart{three}
%         2345678901
%         \nodepart{four}
%         3456789012
%         \nodepart{five}
%         4567890123
%         \nodepart{six}
%         5678901234
%         \nodepart{seven}
%         6789012345
%         \nodepart{eight}
%       };

%       \node[
%       draw,
%       fill = ForestGreen,
%       text = White,
%       left = 3cm of dram,
%       minimum width = 2cm,
%       minimum height = 2cm,
%       ] (cpu) {
%         CPU
%       };

%       \node[
%       below = .5 of dram,
%       text = Black,
%       fill = \bufcolor,
%       draw,
%       minimum width = 3cm,
%       ] (buf) {
%         \only<1>{row buffer}%
%         \only<2>{1234567890}%
%         \only<3>{1234567890}%
%         \only<4>{2345678901}%
%         \only<5>{1234567890}%
%       };

%       \node<1>[
%       right = of dram.north east,
%       ] (receiver) {
%         Получатель
%       };

%       \node<1>[
%       right = of dram.east,
%       ] (sender) {
%         Отправитель
%       };

%       \draw<1> (receiver) -- (dram.one east);
%       \draw<1> (receiver) -- (dram.two east);
%       \draw<1> (sender) -- (dram.three east);
%       \draw<1> (sender) -- (dram.four east);

%       \draw<2> (dram.two east) -- ++(1, 0) |- node[right] {Открытие} (buf);
%       \draw<2> (buf) -- node[below, sloped] {Медленно!} (cpu);

%       \draw<3> (buf) -- node[below, sloped] {Быстро!} (cpu);

%       \draw<4> (dram.three east) -- ++(1, 0) |- node[right] {Открытие} (buf);
%       \draw<4> (buf) -- node[below, sloped] {} (cpu);

%       \draw<5> (dram.two east) -- ++(1, 0) |- node[right] {Открытие} (buf);
%       \draw<5> (buf) -- node[below, sloped] {Медленно!} (cpu);

%     \end{tikzpicture}
%     \only<1>{\caption{Отправитель и получатель используют один и тот же банк
%         памяти}}

%     \only<2>{\caption{Получатель проверяет время доступа к своей памяти}}

%     \only<3>{\caption{Повторное чтение своей памяти будет происходить быстрее}}

%     \only<4>{\caption{Отправитель получает доступ к своей памяти}}

%     \only<5>{\caption{Получатель при своей следующей попытке чтения памяти
%         получит промах строки, соответственно большее время ожидание}}
%   \end{figure}

%   \note<1>{

%     В случае атаки при \textbf{промахе строки}, получатель и отправитель
%     пользуются \textbf{разделяемым банком}, но не строкой.

%   }

%   \note<2>{

%     Получатель постоянно открывает одну и ту же строку в банке. Как только
%     отправитель открывает какую-либо другую строку, то у получателя возрастает
%     задержка во время открытия своей строки.

%   }
%   \note<5>{

%     Таким образом налаживается связь: если в определённый промежуток времени не
%     было промаха строки, то это 0, если промах случился, то 1.

%   }
% \end{frame}

% \subsubsection{Тепловой канал}
% \begin{frame}{\insertsubsubsection}

%   \begin{figure}[h]

%     \begin{tikzpicture}[
%       align = center,
%       ->,
%       > = Stealth,
%       thick,
%       block/.style = {
%         rectangle,
%         draw,
%         fill = ForestGreen,
%         text = White,
%       },
%       ]

%       \node[
%       block,
%       minimum width = 2cm,
%       minimum height = 2cm,
%       label = above:Core 0,
%       ] (core_0) {
%         t$^\circ$C
%       };

%       \node[
%       below = of core_0,
%       block,
%       minimum width = 2cm,
%       minimum height = 2cm,
%       label = above:Core 1,
%       ] (core_1) {
%         \only<1>{\color{red}73$^\circ$C}%
%         \only<2>{49$^\circ$C}%
%         \only<3>{\color{red}66$^\circ$C}%
%         \only<4>{33$^\circ$C}%
%         \only<5>{\color{red}81$^\circ$C}%
%         \only<6>{27$^\circ$C}%
%         \only<7>{31$^\circ$C}%
%         \only<8>{t$^\circ$C}%
%       };

%       \draw<1> (core_1) -- ++(2, 0) |- node[right, text = Black] {\color{ForestGreen}1} (core_0);
%       \draw<2> (core_1) -- ++(2, 0) |- node[right, text = Black] {1\color{ForestGreen}0} (core_0);
%       \draw<3> (core_1) -- ++(2, 0) |- node[right, text = Black] {10\color{ForestGreen}1} (core_0);
%       \draw<4> (core_1) -- ++(2, 0) |- node[right, text = Black] {101\color{ForestGreen}0} (core_0);
%       \draw<5> (core_1) -- ++(2, 0) |- node[right, text = Black] {1010\color{ForestGreen}1} (core_0);
%       \draw<6> (core_1) -- ++(2, 0) |- node[right, text = Black] {10101\color{ForestGreen}0} (core_0);
%       \draw<7> (core_1) -- ++(2, 0) |- node[right, text = Black] {101010\color{ForestGreen}0} (core_0);
%       \draw<8> (core_1) -- ++(2, 0) |- node[right, text = Black] {1010100} (core_0);


%     \end{tikzpicture}
%     \caption{Скрытый канал на основе теплового следа}
%   \end{figure}

%   \note<1>{

%     В 2015 году --- эффективно работает даже при \textbf{наличии жёстких
%       контрмер}: \textbf{псевдоизоляции адресного пространства} критических
%     процессов, либо посредством \textbf{выделения памяти случайным образом} и
%     т. п. Рассматриваемая методика --- \textbf{анализ тепловой активности
%       микропроцессорной системы}.

%     Здесь используются следующие два эффекта. \textbf{Во-первых, остаточные
%       тепловые следы в ядре} сохраняются, даже когда процесс прекратил своё
%     исполнение, и этот \textbf{след частично передаётся следующему процессу по
%       расписанию}. Во-вторых, тепловой след \textbf{влияет и на другие ядра}
%     (если эти несколько ядер на одном чипе расположены).

%   }

%   \note<8>{

%     Так, пропускная способность «теплового канала» на сервере Intel Xeon (с
%     двумя процессорами по восемь ядер) составляет 12,5 бит/с. Номер кредитной
%     карты по такому каналу передаётся за промежуток от пяти секунд до четырёх
%     минут.

%     Также в 2015 году в рамках демонстрации возможностей скрытого теплового
%     канала было показано, что эта методика позволяет \textbf{идентифицировать
%       приложения на основе их тепловых следов}. При этом с течением времени
%     эффективность и \textbf{производительность скрытых тепловых каналов будет
%       только расти}, потому что пользователю предоставляется все более подробная
%     информация о температуре системы --- чтобы предпринимать \textbf{эффективные
%       меры для охлаждения}.

%   }

% \end{frame}
