\begin{frame}{\insertsubsection}

  CVE-2017-5715: тренировка предсказателя переходов

  \begin{figure}[h]
    \includegraphics[height = .7\textheight]{spectre_logo}
    \caption{Spectre}
  \end{figure}

  \note{


  }
\end{frame}

\subsubsection{Предсказатель переходов и его тренировка}
\begin{frame}{\insertsubsubsection}

  \begin{figure}[h]
    \begin{tikzpicture}[
      align = center,
      ->,
      > = Stealth,
      thick,
      ampersand replacement = \&,
      block/.style = {
        rectangle,
        draw,
        fill = ForestGreen,
        text = White,
        text centered,
      },
      ]

      \node[
      text = Black,
      font = \ttfamily,
      ] (code) {
        Animal* a = \color{ForestGreen}\only<1-4>{bird}\only<5->{fish};
      };

      \node[
      below = .5 of code,
      text = Black,
      font = \ttfamily,
      ] (move) {
        a->move()
      };

      \node[
      below left = 3 of move,
      text = Black,
      font = \ttfamily,
      minimum width = 3cm,
      ] (cache) {
        arr1[untrusted]
      };

      \node[
      circle,
      below = of move,
      text = White,
      fill = ForestGreen,
      draw,
      label = below:Предсказание,
      ] (zero) {
        \only<1-3>{swim()}%
        \only<4->{fly()}%
      };

      \node[
      below right = 3 of move,
      text = Black,
      font = \ttfamily,
      minimum width = 3cm,
      ] (zero) {
        0
      };

      \draw<1-2,4-5,7> (move) -> node[above, sloped] {fly()} (cache);
      \draw<1,3,4-6> (move) -> node[above, sloped] {swim()} (zero);

      \draw<2>[ForestGreen] (move) -> node[above, sloped] {swim()} node[below, sloped] {Спекулятивно} (zero);
      \draw<3>[ForestGreen] (move) -> node[above, sloped] {fly()} node[below, sloped] {Реально} (cache);

      \draw<6>[ForestGreen] (move) -> node[above, sloped] {fly()} node[below, sloped] {Спекулятивно} (cache);
      \draw<7>[ForestGreen] (move) -> node[above, sloped] {swim()} node[below, sloped] {Реально} (zero);

    \end{tikzpicture}
    % \caption{}

  \end{figure}

  \note<1>{

    Ранее уже рассказывалось про предсказатель переходов, а также про буфер.

    Для атаки требуется досконально знать, \textbf{как работает предсказатель
      переходов}.

  }

  \note<6>{

    \textbf{Внимание}, спекулятивно выполняется натренированная нами ветка! В
    итоге будет исполняться код, который совершит атаку на кэш.

  }
\end{frame}

\begin{frame}{\insertsubsubsection}

  \begin{figure}[h]
    \begin{tikzpicture}[
      align = center,
      ->,
      > = Stealth,
      thick,
      ampersand replacement = \&,
      bp/.style = {
        rectangle,
        rectangle split,
        rectangle split parts = 2,
        rectangle split horizontal,
        rectangle split part fill = {
          ForestGreen, NavyBlue
        },
        draw,
        text = White,
      },
      ]

      \node[
      bp,
      ] (bp) {
        \color{Black}0xEBE45A82
        \nodepart{two}
        T,T,N,N,T,T,N,N
        \nodepart{three}
      };

      \node[
      bp,
      above left = of bp.center,
      label = above:Процесс A,
      ] (proc_a) {
        0x0000 \color{Black}EBE45A82
        \nodepart{two}
        Переход A
        \nodepart{three}
      };

      \node[
      bp,
      above right = of bp.center,
      label = above:Процесс B (ядро/гипервизор),
      ] (proc_b) {
        0xFFFF \color{Black}EBE45A82
        \nodepart{two}
        Переход B
        \nodepart{three}
      };

      \draw (proc_a.one south) |- (bp.one west);
      \draw (proc_b.one west) -> (bp.one north);


    \end{tikzpicture}
    \caption{В BTB используются виртуальные адреса, а также возникают коллизии}

  \end{figure}


  \note{

    Из-за того, что возникают коллизии в таблице BTB, мы можем
    \textbf{натренировать} его таким образом, чтобы спекулятивно исполнялся
    нужный нам переход.

    Существует \textbf{не один способ} тренировки предсказателя переходов.

    Также существует возможность создания ROP цепочки из \textbf{гаджетов
      программы-жертвы} и натренировать на него, но для этого требуется знать
    адрес. Чтобы \textbf{узнать адрес перехода} можно применить \textbf{атаку по
      сторонним каналам} на предсказатель переходов.

  }
\end{frame}

\subsubsection{И снова спекулятивное выполнение}
\begin{frame}[fragile]{\insertsubsubsection}

  \begin{minted}{c}
  if (untrusted_offset_from_user < array1_size)
    y = array2[((array1[untrusted_offset_from_user] & 1) * 0x100) + 0x200];
  \end{minted}

  \note{

    Ничего нового, используется всё тот же код, чаще всего ROP цепочка, которая
    приводит к атакам на кэш.

  }
\end{frame}

\subsubsection{Предотвращение}
\begin{frame}{\insertsubsubsection}

  \LARGE

  Частично такое же, как и в случае с Variant 1

  \begin{itemize}
  \item отключение предсказателя переходов (Indirect Branch Restrict
    Speculation)
  \item очистка буфера предсказателя переходов при переключении контекста
    (Indirect Branch Predictor Barrier)
  \item выключение/включение MMU
  \item retpoline --- «оборачивание» косвенных переходов
  \end{itemize}

  \note{

    Всё медленно!

    retpoline --- замена всех косвенных переходов, дополнение инструкций
    возврата, паузы перед непрямыми вызовами функций

  }

\end{frame}