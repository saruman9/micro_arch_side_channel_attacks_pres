\subsubsection{How DRAM works}
\begin{frame}{\insertsubsubsection}

  \newcommand*\dramcolor{dWhite, dWhite, dWhite, dWhite, dWhite, dWhite, dWhite}

  \only<2>{
    \renewcommand*\dramcolor{dWhite, dWhite, dGreen, dWhite, dWhite, dWhite,
      dWhite}
  }

  \begin{figure}[h]
    \begin{tikzpicture}[
      align = center,
      ->,
      > = Stealth,
      barstyle/.style 2 args = {%
        draw = dBlue,
        minimum width = 2em,
        fit = {(####1.south) (####1.north)},
        inner ysep = 0pt,
        label = {[rotate = 90]center:####2}
      },
      ]

      \node[
      draw = dGreen,
      minimum width = 2cm,
      ] (proc) {CPU};


      \node[
      draw = dGreen,
      right = 3 of proc,
      ] (mmu) {Memory controller};

      \node[
      draw = dGreen,
      right = 3 of mmu,
      rectangle split,
      rectangle split parts = 7,
      rectangle split part fill = {\dramcolor},
      minimum width = 2cm,
      ] (dram) {%
        Row 0%
        \nodepart{two}%
        Row 1%
        \nodepart{three}%
        Row 2%
        \nodepart{four}%
        Row 3%
        \nodepart{five}%
        Row 4%
        \nodepart{six}%
        Row 5%
        \nodepart{seven}%
        Row 6%
        \nodepart{seven}%
        Row 7%
      };

      \node[
      barstyle={dram}{%
        \only<1>{Row buffer}%
        \only<2>{Row 2}%
      },
      left = 0 of dram,
      ] (r_buf) {};


      \node[
      fit = (dram) (r_buf),
      label = above:DRAM array,
      ] {};

      \draw (proc) -- node[above] {Memory bus} (mmu);
      \draw (mmu) -- node[above] {Request} (r_buf);

    \end{tikzpicture}
    \caption{A simple computer system with a single DRAM
      array}\label{dram_example}
  \end{figure}

  \note<1>{

    \textbf{DRAM (dynamic random-access memory, динамическая память с
      произвольным доступом).}

    DRAM имеет большую задержку в сравнении с кэш-памятью. Причина большой
    задержки не только в том, что ячейки DRAM имеют меньшую тактовую частоту, но
    и в том, как DRAM организован и подключён к процессору. Современные
    процессоры используют чипы-контроллеры памяти, которые позволяют
    передавать/получать данные в/с DRAM.


    DRAM содержит: \textbf{строки (row)} и \textbf{колонки (columns)} (обычно
    1024).

  }

  \note<2>{

    Строка может быть \textbf{открытой} и \textbf{закрытой}. Если какая-либо
    строка открыта, то она вся сохраняется в \textbf{буфер строки (row buffer)}.

    Если текущая открытая строка содержит необходимые данные, то контроллер
    памяти просто берёт их из буфера строки. Эта ситуация очень похожа на кэш
    попадание и называется \textbf{попадание строки}. Если текущая открытая
    строка не содержит нужных данных, то это называется \textbf{промах строки}.
    \textbf{Контроллер памяти} в таком случае сначала \textbf{закрывает строку},
    т. е. \textbf{записывает буфер строки обратно в DRAM}, а затем
    \textbf{открывает нужную строку} и считывает данные из буфера строки. Также
    как и промахи кэша, промахи строки вызывают повышение задержки.

  }

\end{frame}

\subsubsection{DRAM organization}
\begin{frame}{\insertsubsubsection}

  \begin{figure}[h]
    \begin{tikzpicture}[
      align = center,
      ->,
      > = Stealth,
      bank/.style = {
        draw = dGreen,
        rectangle split,
        rectangle split parts = 3,
      },
      ]

      \node<1>[inner sep = 0, label = above:DIMM] (dram) {%
        \includegraphics[height = .3\textheight]{DRAM_ddr4}%
      };

      \node<2-4>[inner sep = 0] (dram_0) {%
        \includegraphics[height = .2\textheight]{DRAM_ddr4}%
      };

      \node<2-4>[inner sep = 0, below = of dram_0] (dram_1) {%
        \includegraphics[height = .2\textheight]{DRAM_ddr4}%
      };

      \node<2-4>[
      inner sep = 0,
      left = 2cm of $(dram_0.west)!.5!(dram_1.west)$,
      ] (cpu) {\includegraphics[height = .3\textheight]{cpu}};

      \draw<2-4> (cpu) -- ++(2, 0) -- ++(0, 1)  |- node[above] {Channel 0} (dram_0.west);
      \draw<2-4> (cpu) -- ++(2, 0) -- ++(0, -1) |- node[below] {Channel 1} (dram_1.west);

      \node<3-4>[
      right = 0.3 of dram_0,
      ] (rank_0) {Front of DIMM:\\rank 0};

      \draw<3-4> (dram_0) to [out = 80, in = 60, looseness = 8]%
          node[above] {Back of DIMM: rank 1} (dram_0);
      \draw<3-4> (dram_0) -> (rank_0);

      \draw<4>[dRed,
      rounded corners,
      very thick,
      ] (2.10, -2.7) rectangle (2.67, -2.0);

      \node<4> (chip_capt) at (3.5, -2.4) {Chip};

      \node<5>[bank,
      ] (bank_0) {%
        Row 0%
        \nodepart{two}%
        $\cdots$%
        \nodepart{three}%
        Row 32767%
      };

      \node<5>[
      rectangle,
      draw = dBlue,
      below = .1 of bank_0,
      ] (r_buf_0) {Row buffer};

      \node<5>[
      fit = (bank_0) (r_buf_0),
      draw,
      label = above:Bank 0,
      ] {};

      \node<5>[bank,
      right = of bank_0,
      ] (bank_1) {%
        Row 0%
        \nodepart{two}%
        $\cdots$%
        \nodepart{three}%
        Row 32767%
      };

      \node<5>[
      rectangle,
      draw = dBlue,
      below = .1 of bank_1,
      ] (r_buf_1) {Row buffer};

      \node<5>[
      fit = (bank_1) (r_buf_1),
      draw,
      label = above:Bank 1,
      ] {};

      \node<5>[bank,
      right = of bank_1,
      ] (bank_n) {%
        Row 0%
        \nodepart{two}%
        $\cdots$%
        \nodepart{three}%
        Row 32767%
      };

      \node<5>[
      rectangle,
      draw = dBlue,
      below = .1 of bank_n,
      ] (r_buf_n) {Row buffer};

      \node<5>[
      fit = (bank_n) (r_buf_n),
      draw,
      label = above:Bank n,
      ] {};

      \node<5>[
      rectangle,
      draw = dGreen,
      minimum width = 2cm,
      minimum height = 2cm,
      above = of bank_1,
      label = above:Chip,
      ] (chip) {%
        Bank 0\\%
        Bank 1\\%
        $\cdots$\\%
        Bank n%
      };

      \node<6>[
      rectangle split,
      rectangle split parts = 5,
      rectangle split horizontal,
      draw = dGreen,
      label = above:Row of DRAM,
      ] (dram_row) {%
        Cell 0%
        \nodepart{two}%
        Cell 1%
        \nodepart{three}%
        Cell 2%
        \nodepart{four}%
        $\cdots$%
        \nodepart{five}%
        Cell n%
      };

      \node<6>[
      below = of dram_row,
      ] (cap) {Capacitor};

      \draw<6> (cap) -> (dram_row.one south);
      \draw<6> (cap) -> (dram_row.two south);
      \draw<6> (cap) -> (dram_row.three south);
      \draw<6> (cap) -> (dram_row.five south);

    \end{tikzpicture}
    % \caption{DRAM organization}\label{dram_arch}
  \end{figure}

  \note<1>{

    Для повышения производительности работы с DRAM были использованы те же
    методы, что и в случае с кэшем. Современные компьютерные системы
    организовывают DRAM в виде \textbf{каналов}, \textbf{DIMM (Dual Inline
      Memory Modules)}, \textbf{рангов} и \textbf{банков}.

    Количество памяти увеличивается в разы при перемножении количества банков на
    количество \textbf{DIMM модулей}.

  }

  \note<2>{

    Для \textbf{увеличения ширины потока данных} и увеличения количества
    параллельных потоков современные компьютерные системы используют
    \textbf{несколько каналов}. Каждый канал управляется независимо и
    параллельно через DRAM шину.

  }

  \note<3>{

    Современные модели DRAM имеют обычно от \textbf{1 до 4 рангов}, количество
    которых увеличивается при перемножении с количеством банков. Такого рода
    параллелизм позволяет \textbf{уменьшить промах строки}.

  }

  \note<5>{

    Каждый из этих банков имеет своё собственное состояние и может иметь
    \textbf{независимо от других банков} открытую строку.

    Современная DDR3 DRAM память имеет 8 банков, а DDR4 DRAM 16 банков (на
    ранг).

  }

  \note<6> {

    В случае если два адреса \textbf{отображаются на пространство одного и того
      же} DIMM модуля, ранга и банка, то эти адреса физически
    \textbf{расположены рядом друг с другом} в DRAM. В таких случаях два адреса
    оказываются в банке с одним и тем же номером. В случае, если адреса
    отображаются на пространство банков с одним и тем же номером, но разных
    рангов или DIMM'ов, то они не расположены физически рядом.

    Также как и с кэшем, существуют функции, которые производят отображение
    физического адреса на конкретные канал, DIMM, ранг и банк. Как работают эти
    функции, публично известно только у AMD, Intel не публиковала никакой
    информации. Однако, относительно недавно (2015, 2016) был произведён
    реверс-инжиниринг данных функций. Знание того, \textbf{как производится
      отображение физического адреса на физические структуры} даёт нам новый
    вектор атак --- \textbf{атаки по сторонним каналам на DRAM}.

  }

\end{frame}