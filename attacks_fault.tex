\begin{frame}{\insertsection}

  \begin{figure}[h]
    \center%
    \includegraphics[height = .7\paperheight]{hot_cpu}
    \caption{Software-based microarchitectural fault attacks do not require
      physical access, but instead only some form of code execution on the
      target system}
  \end{figure}

  \note{

    Обычно предполагается, что безопасность системы и программная безопасность
    опирается на безопасность аппаратную и на то, что в аппаратном средстве нет
    ошибок. Однако, это не так, и \textbf{аппаратные средства не идеальны},
    особенно часто дефекты встречаются в случаях, когда работа производится
    \textbf{за границами спецификации}.

    Уникальность атак, основанных на дефектах микроархитектуры в том, что они
    используют эффекты, вызванные микроархитектурными элементами или операциями,
    которые реализованы на микроархитектурном уровне. В атаках, которые основаны
    на использовании программного обеспечения все микроархитектурные эффекты и
    операции вызываются из программного обеспечения.

  }

\end{frame}
