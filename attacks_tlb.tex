\begin{frame}{\insertsubsection}

  \begin{figure}[h]

    \begin{tikzpicture}[
      align = center,
      ->,
      > = Stealth,
      ampersand replacement = \&,
      block/.style = {
        draw = dGreen,
      },
      ]

      \node[
      block,
      ] (tlb) {TLB};

      \node[
      left = of tlb,
      ] (virt_addr) {Virtual\\ Address};

      \node[
      below = of tlb,
      ] (hit) {Mapped address\\ returns quicker!};

      \node[
      block,
      label = above:\color{dBlue}Translation table,
      right = 6cm of tlb,
      rectangle split,
      rectangle split parts = 5,
      rectangle split empty part width = 1cm,
      ] (table) {%
        PML4E%
        \nodepart{two}%
        PDPTE%
        \nodepart{three}%
        PDE%
        \nodepart{four}%
        PTE%
        \nodepart{five}%
        Physical address%
      };

      \draw (virt_addr) -> (tlb);
      \draw (tlb) -> node[right] {\color{dGreen}Hit} (hit);

      \draw (tlb) -> node[below] {\color{dRed}Miss}
        node[above] {Unmapped address\\takes $\approx$ 40 cycles\\%
        more for page table walk} (table);

    \end{tikzpicture}
    \caption{A translation lookaside buffer (TLB) is a memory cache that is used
      to reduce the time taken to access a user memory
      location}\label{tlb_example}
  \end{figure}

  \note{

    Трансляция адресов должна происходить очень быстро. С использованием таблиц
    трансляций, расположенных в памяти, данная операция быстро выполняться не
    может. По этой причине был введён кеш для трансляции адресов, который
    помогает уменьшить задержку при процессе трансляции --- \textbf{буфер
      ассоциативной трансляции (translation lookaside buffer, TLB)}.

    Атака впервые была представлена Ralf Hund \textbf{в 2013 году}. Попытка
    чтения или записи памяти, к которой нет доступа по причине того, что данный
    участок памяти \textbf{используется ядром, занимает меньше времени}, если бы
    память не была размечена вовсе, т. к. используемые адреса памяти попадают в
    кэш независимо от уровня привилегий.

    Это позволяет узнать, какие адреса используются, и более того, узнать какие
    адреса \textbf{используются той или иной частью ядра}, т. е. данный вид
    атаки позволяет обойти технику рандомизации памяти в ядерном пространстве
    (kernel address-space-layout randomization, \textbf{KASLR}).

  }
\end{frame}
