\begin{frame}{\insertsubsection}

  \begin{figure}[h]

    \begin{tikzpicture}[
      align = center,
      ->,
      > = Stealth,
      thick,
      ampersand replacement = \&,
      block/.style = {
        draw,
        fill = ForestGreen,
        text = White,
      },
      ]

      \node[
      block,
      ] (tlb) {
        TLB
      };

      \node[
      left = of tlb,
      text = Black,
      ] (virt_addr) {
        Виртуальный\\
        адрес
      };

      \node[
      below = of tlb,
      text = Black,
      ] (hit) {
        Использованный раннее адрес\\
        вернётся быстрее
      };

      \node[
      block,
      label = above:Таблицы трансляции,
      right = 6cm of tlb,
      rectangle split,
      rectangle split parts = 5,
      rectangle split empty part width = 1cm,
      ] (table) {
        PML4E
        \nodepart{two}
        PDPTE
        \nodepart{three}
        PDE
        \nodepart{four}
        PTE
        \nodepart{five}
        Физический адрес
        \nodepart{six}
      };

      \draw (virt_addr) -> (tlb);
      \draw (tlb) -> node[right] {Hit} (hit);

      \draw (tlb) -> node[below] {Miss}
      node[above, text = Black] {Для прохождения всей\\таблицы трансляции\\
        необходимо дополнительно\\$\approx40$ тактов} (table);

    \end{tikzpicture}
    \caption{Translation lookaside buffer (TLB) используется как для ускорения
      трансляции виртуальных адресов ядерного пространства, так и
      пользовательского!}\label{tlb_example}
  \end{figure}

  \note{

    Трансляция адресов должна происходить очень быстро. С использованием таблиц
    трансляций, расположенных в памяти, данная операция быстро выполняться не
    может. По этой причине был введён кеш для трансляции адресов, который
    помогает уменьшить задержку при процессе трансляции --- \textbf{буфер
      ассоциативной трансляции (translation lookaside buffer, TLB)}.

    Атака впервые была представлена Ralf Hund \textbf{в 2013 году}. Попытка
    чтения или записи памяти, к которой нет доступа по причине того, что данный
    участок памяти \textbf{используется ядром, занимает меньше времени}, если бы
    память не была размечена вовсе, т. к. используемые адреса памяти попадают в
    кэш независимо от уровня привилегий.

    Это позволяет узнать, какие адреса используются, и более того, узнать какие
    адреса \textbf{используются той или иной частью ядра}, т. е. данный вид
    атаки позволяет обойти технику рандомизации памяти в ядерном пространстве
    (kernel address-space-layout randomization, \textbf{KASLR}).

  }
\end{frame}
