\subsubsection{Алгоритм работы DRAM}
\begin{frame}{\insertsubsubsection}

  \begin{figure}[h]

    \newcommand*\dramcolor{White, White, White,
      White, White, White, White}

    \newcommand*\bufcolor{White}

    \only<2>{
      \renewcommand*\dramcolor{White, ForestGreen, White,
      White, White, White, White}
    }

    \only<3-5>{
      \renewcommand*\dramcolor{White, ForestGreen, White,
        White, White, White, White}
      \renewcommand*\bufcolor{ForestGreen}
    }

    \begin{tikzpicture}[
      align = center,
      ->,
      > = Stealth,
      thick,
      dram/.style = {
        rectangle,
        rectangle split,
        rectangle split parts = 7,
        rectangle split part fill = \dramcolor,
        minimum width = 3cm,
        draw,
        text = Black,
        text centered,
      },
      block/.style = {
        rectangle,
        draw,
        fill = ForestGreen,
        text = White,
        text centered,
      },
      ]

      \node[
      dram,
      label = above:DRAM банк,
      ] (dram) {
        0123456789
        \nodepart{two}
        1234567890
        \nodepart{three}
        2345678901
        \nodepart{four}
        3456789012
        \nodepart{five}
        4567890123
        \nodepart{six}
        5678901234
        \nodepart{seven}
        6789012345
        \nodepart{eight}
      };

      \node[
      draw,
      fill = ForestGreen,
      text = White,
      left = 3cm of dram,
      minimum width = 2cm,
      minimum height = 2cm,
      ] (cpu) {
        CPU
      };

      \node[
      below = .5 of dram,
      text = Black,
      fill = \bufcolor,
      draw,
      minimum width = 3cm,
      ] (buf) {
        \only<1-2>{row buffer}%
        \only<3-5>{1234567890}%
      };

      \draw<2, 5> (cpu) -- node[sloped, above] {Чтение} (dram.two west);

      \draw<3> (dram.two east) -- ++(1, 0) |- node[right] {Открытие} (buf);

      \draw<4> (buf) -- (cpu);

      \draw<5> (buf) -- node[sloped, above] {Быстрее!} (cpu);


    \end{tikzpicture}
    \only<1>{\caption{Работа DRAM (ещё раз)}}\label{fig:draw_row_buffer}

    \only<2>{\caption{CPU запрашивает на чтение строку №1}}

    \only<3>{\caption{DRAM открывает строку №1}}

    \only<4>{\caption{CPU читает строку №1 из буфера строки}}

    \only<5>{\caption{CPU снова запрашивает на чтение строку №1, которая уже
        есть в буфере строки, чтение происходит быстрее}}
  \end{figure}

  \note<1>{

    Ещё раз о том, как работает \textbf{row buffer}.

  }

  \note<5>{

    При попадании строки чтение данных происходит быстрее, при промахе строки
    --- медленнее, что похоже на поведение при работе с кэшем.

  }

\end{frame}

\subsubsection{Типы атак на DRAM}
\begin{frame}{\insertsubsubsection}
  \begin{itemize}
  \item DRAMA
  \item Row hit (Flush + Reload)
  \item Row miss (Prime + Probe)
  \item и др.
  \end{itemize}

  \note{

    Существуют атаки непосредственно на DRAM (DRAM addressing, \textbf{DRAMA}).

    Атака при \textbf{промахе строки} похожа на атаку на кэш \textbf{Prime +
      Probe}, атака при \textbf{попадании строки} сравнима с атакой
    \textbf{Flush + Reload}. Оба типа атаки работают и при \textbf{отсутствии
      разделяемой памяти}. DRAMA эксплуатирует буфер строки DRAM, как будто это
    \textbf{кэш с прямым отображением используемый банком}.

    Во время подготовки атаки DRAMA применяли методы реверс-инжиниринга,
    основанные на подобном поведении DRAM, для того, чтобы \textbf{выявить
      местоположение и архитектуру банков}.
    
  }
\end{frame}

% \subsubsection{Row hit атака (Flush + Reload)}
% \begin{frame}{\insertsubsubsection}

%   \begin{figure}[h]

%     \newcommand*\dramcolor{Green, NavyBlue, ForestGreen, ForestGreen, White,
%       White, White}

%     \newcommand*\bufcolor{Green}

%     \only<2>{
%       \renewcommand*\bufcolor{NavyBlue}
%     }

%     \only<3>{
%       \renewcommand*\bufcolor{Green}
%     }

%     \only<4>{
%       \renewcommand*\bufcolor{NavyBlue}
%     }

%     \begin{tikzpicture}[
%       align = center,
%       ->,
%       > = Stealth,
%       thick,
%       dram/.style = {
%         rectangle,
%         rectangle split,
%         rectangle split parts = 7,
%         rectangle split part fill = \dramcolor,
%         minimum width = 3cm,
%         draw,
%         text = Black,
%         text centered,
%       },
%       block/.style = {
%         rectangle,
%         draw,
%         fill = ForestGreen,
%         text = White,
%         text centered,
%       },
%       ]

%       \node[
%       dram,
%       label = above:DRAM банк,
%       ] (dram) {
%         0123456789
%         \nodepart{two}
%         1234567890
%         \nodepart{three}
%         2345678901
%         \nodepart{four}
%         3456789012
%         \nodepart{five}
%         4567890123
%         \nodepart{six}
%         5678901234
%         \nodepart{seven}
%         6789012345
%         \nodepart{eight}
%       };

%       \node[
%       draw,
%       fill = ForestGreen,
%       text = White,
%       left = 3cm of dram,
%       minimum width = 2cm,
%       minimum height = 2cm,
%       ] (cpu) {
%         CPU
%       };

%       \node[
%       below = .5 of dram,
%       text = Black,
%       fill = \bufcolor,
%       draw,
%       minimum width = 3cm,
%       ] (buf) {
%         \only<1>{0123456789}%
%         \only<2>{1234567890}%
%         \only<3>{0123456789}%
%         \only<4>{1234567890}%
%       };

%       \draw<1> (dram.one east) -- ++(1, 0) |- node[right] {Открытие} (buf);
%       \draw<1> (buf) -- node[below, sloped] {Медленно!} (cpu);

%       \draw<2> (dram.two east) -- ++(1, 0) |- node[right] {Открытие} (buf);
%       \draw<2> (buf) -- node[below, sloped] {Медленно!} (cpu);

%       \draw<3> (dram.one east) -- ++(1, 0) |- node[right] {Открытие} (buf);

%       \draw<4> (dram.two east) -- ++(1, 0) |- node[right] {Открытие} (buf);
%       \draw<4> (buf) -- node[below, sloped] {Быстро\\при повторном\\обращении!} (cpu);

%     \end{tikzpicture}
%     \only<1>{\caption{Атакующий запрашивает строку №0, содержимое которой
%         принадлежит атакующему}}

%     \only<2>{\caption{Атакующий запрашивает строку №1, содержимое которой
%         частично принадлежит и атакующему, и жертве}}

%     \only<3>{\caption{Сбросим (вытесним) буфер и передадим управление
%         программе--жертве}}

%     \only<4>{\caption{В случае, если жертва обращалась к данному адресу, то это
%         можно вычислить по времени повторного обращения}}
%   \end{figure}

%   \note<1>{

%     В случае атаки при \textbf{попадании строки}, атакующий и жертва используют
%     \textbf{одну и ту же строку} DRAM.

%   }

%   \note<3>{

%     Атакующий загружает другую строку в буфер строки (тем самым закрывая
%     открытую строку), что похоже на начало атаки \textbf{Flush + Reload}.

%   }

%   \note<4>{

%     Если жертва \textbf{обращалась} к общей уже закрытой строке, то данная
%     строка снова будет загружена в буфер строки, что выяснится позже при
%     повторном обращении атакующего к этой строке (\textbf{время обращения будет
%       меньше}).

%   }

% \end{frame}