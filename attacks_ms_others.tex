\begin{frame}{\insertsubsection}

  Spectre-NG
  \begin{itemize}
  \item MeltdownPrime \& SpectrePrime
  \item SgxPectre
  \item SMM Speculative Execution Attacks
  \item BranchScope
  \item LazyFP
  \item ...
  \end{itemize}

  \note{

    \textbf{OpenBSD} --- LazyFP, отключение Hyper-Threading (TLBleed).

    \begin{itemize}
    \item \textbf{другой способ атаки на кэш} + задействование двух ядер CPU ---
      утечка данных работы \textbf{протокола согласования содержимого кэша для
        разных ядер CPU} (Invalidation-Based Coherence Protocol
    \item позволяет обойти средства изоляции кода и данных, предоставляемые
      технологией \textbf{Intel SGX} (Software Guard Extensions)

    \item SMM --- \textbf{режим системного управления} --- запускается
      специальная программа в привилегированном режиме.

    \item \textbf{variant 2} + вместо BTB --- \textbf{направления ветвления для
        спекулятивного перехода (directional branch predictor)} и манипулирует
      содержимым \textbf{таблицы с историей шаблонов переходов (PHT, Pattern
        History Table)}.
    \item \textbf{variant 3a} --- «ленивое» режим \textbf{переключения контекста
        FPU}, при котором реальное восстановление состояния регистров
      производится \textbf{не сразу} после переключения контекста, а только при
      выполнении первой инструкции
    \end{itemize}

  }

\end{frame}

\begin{frame}{\insertsubsection}

  \begin{itemize}
  \item TLBleed
  \item Spectre 1.1, 1.2 (Speculative Buffer Overflows)
  \item SpectreRSB
  \item NetSpectre
  \item ...
  \end{itemize}

  TotalMeltdown?

  \note{

    \begin{itemize}
    \item ML атака на TLB, с помощью Hyper-Threading технологии (Intel, AMD).
    \item Spectre 1 + Buffer Overflows, спекулятивная запись.
    \item Засорение RSB (return stack buffer), спекулятивное выполнение после
      return.
    \item Производится \textbf{поиск leak и transmit гаджетов}, низкая
      производительность (1-3 байта за 3-8 часов).
    \end{itemize}

  }

\end{frame}