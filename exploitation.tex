\begin{frame}{\insertsubsection}

  \only<1-2, 4-5, 7-8, 10-11>{

    \begin{columns}

    \column{.5\textwidth}

    {\color{dBlue}Four components of speculative techniques}

    \begin{enumerate}
    \only<1->{\item<alert@1-2> Speculation primitive}%
    \only<4->{\item<alert@4-5> Windowing gadget}%
    \only<7->{\item<alert@7-8> Disclosure gadget}%
    \only<10->{\item<alert@10-11> Disclosure primitive}%
    \end{enumerate}
  }

  \note<1>{

    А так ли всё легко и просто? Представим, что мы хотим совершить атаку.

    Рассмотрим на примере Spectre v2 (variant 2).

    Процессор \textbf{должен поддерживать} возможность спекулятивного
    выполнения.

    Требуется \textbf{найти место} в коде, где может происходить спекулятивное
    выполнение по желанию атакующего.

  }

  \note<4>{

    Требуется найти гаджеты, которые позволят создать достаточно
    \textbf{длительное по времени выполнения окно} для спекулятивного
    выполнения.
    
  }

  \note<7>{

    Требуется найти гаджеты, которые позволят \textbf{считать необходимую
      закрытую информацию} в ходе спекулятивного выполнения.
    
  }

  \note<10>{

    Требуется возможность \textbf{считать полученную} через сторонний канал
    информацию или \textbf{удостовериться}, что она была считана.

  }

  \only<3, 6, 9, 12>{
    \begin{figure}[h]
      \begin{tikzpicture}[
        align = center,
        ->,
        > = Stealth,
        block/.style = {
          draw = dGreen,
        },
        ]

        \node[
        block,
        ] (type_bp) {Type of BP};

        \node[
        block,
        right = of type_bp,
        rotate = 3,
        ] (algo_bp) {Algorithm of BP};

        \node[
        block,
        right = of algo_bp,
        rotate = -2,
        ] (work_bp) {Environment of BP};

        \node<6->[
        block,
        above = 0 of $(type_bp.north)!0.5!(algo_bp.north)$,
        rotate = 2,
        ] (search_cache_gadgets) {Search/create gadgets};

        \node<6->[
        block,
        above = .13 of $(work_bp.north)!0.5!(algo_bp.north)$,
        % rotate = 1,
        ] (info_cache) {Contents of cache};

        \node<9->[
        block,
        above = .05 of $(search_cache_gadgets.north)!0.5!(info_cache.north)$,
        rotate = 4,
        ] (bypass_aslr) {Bypass ASLR};

        \node<9->[
        block,
        above = 0 of info_cache.north east,
        rotate = -8,
        ] (bypass_others) {Bypass others techniques};

        \node<12->[
        block,
        above = 1.7 of search_cache_gadgets,
        rotate = 65,
        ] (timers) {High-resolution timer};

        \node<12->[
        block,
        fill = dGreen,
        above = -0.5 of bypass_aslr,
        minimum width = 5cm,
        minimum height = 2cm,
        fill opacity = 0.5,
        text opacity = 1,
        ] (noise) {Noise};

        \node<12->[
        block,
        above = 0 of noise,
        ] (flush_reload) {Flush + Reload};

        \node<12->[
        inner sep = 0,
        right = of bypass_others,
        ] (vovka) {\includegraphics[width = .15\textwidth]{vovka}};

      \end{tikzpicture}
      \only<1-5>{\caption{The foundation of the Speculative-Based Attack tower}}

      \only<6-8>{\caption{The Speculative-Based Attack tower}}

      \only<9->{\caption{The Speculative-Based Attack \textbf{Babel} tower}}
    \end{figure}
  }

  \note<3>{

    Выберем тренировку предсказателя переходов. Мы столкнёмся:

    \begin{itemize}
    \item требуется знать \textbf{вид предсказателя переходов}
    \item требуется знать \textbf{алгоритм работы предсказателя переходов}
    \item для \textbf{разных процессоров --- разные условия}, например, в i7 два
      буфера предсказателя переходов.
    \end{itemize}

    В whitepaper и в PoC даны \textbf{примеры для конкретных процессоров}.

  }

  \note<6>{

    Выберем гаджеты для загрузки некэшированных данных. Мы столкнёмся:

    \begin{itemize}
    \item в случае JIT --- создание гаджетов, в других случаях гаджеты следует
      искать,
    \item что хранится в кэше на данный момент.
    \end{itemize}

    В whitepaper и в PoC даны \textbf{примеры для конкретных процессоров}.

  }

  \note<9>{

    В whitepaper и в PoC \textbf{все защиты отключены}.

  }

  \note<12>{

    Что? Ещё одна атака?

  }

  \only<1-2, 4-5, 7-8, 10-11>{\column{.5\textwidth}}

  \only<2>{

    \begin{itemize}
    \item Bypass out of bounds checks
    \item Training of branch predictor
    \item Speculatively read an earlier value of the data
    \item Pending exceptions
    \item Exploit branch history table
    \item Exploit the Return Stack Buffer
    \item Speculatively write to register (buffer overflow)
    \end{itemize}
    
  }

  \note<2> {

    На данный момент существует несколько техник, позволяющих производить
    спекулятивное выполнение по желанию атакующего.

    Для того, чтобы использовать те или иные техники \textbf{требуется
      досконально знать микроархитектуру процессора}.
    
  }

  \only<5>{

    \begin{itemize}
    \item Non-cached loads
    \item Dependency chain of loads
    \item Dependency chain of integer ALU operations
    \end{itemize}
    
  }

  \note<5>{

    Не \textbf{везде есть такие цепочки} в окружении. В некоторых случаях
    \textbf{требуется знать, какие данные сейчас в кэше}.
    
  }

  \only<8>{

    \begin{itemize}
    \item ASLR
    \item CFI
    \item SMAP
    \item DEP/NX
    \item retpoline
    \item and others.
    \end{itemize}
    
  }

  \note<8>{

    Для применения необходимых гаджетов требуется обойти некоторые системы
    защиты.
    
  }

  \only<11>{

    \begin{itemize}
    \item Architecture of cache
    \item Replacement policies
    \item Exclusive and inclusive
    \item Type of cache attack
    \item Noise
    \item High-resolution timer
    \item and etc.
    \end{itemize}

  }


  \only<1-2, 4-5, 7-8, 10-11>{\end{columns}}

  \note<11>{

    \begin{itemize}
    \item Требуется знать, какой тип кэша используется.
    \item Требуется знать, какие данные сейчас хранятся в кэше, как выталкивать.
    \item На какой кэш будет направлена атака.
    \item Возможность проведения той или иной атаки.
    \item Возможность многократного повторения атаки.
    \item Для измерения времени требуются высокоточные счётчики.
    \end{itemize}

  }

\end{frame}
