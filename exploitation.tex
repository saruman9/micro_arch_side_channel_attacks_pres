\begin{frame}{\insertsection}

  \newcommand{\insm}{%
    \smash{\raisebox{.5\dimexpr1\baselineskip+4\itemsep+2\parskip}{$\left.\rule{0pt}{.5\dimexpr9\baselineskip+3\itemsep+3\parskip}\right\}$\
        \parbox{5.5cm}{Микроархитектура --- ?}}} }


  \only<1-2, 4-5, 7-8, 10-11>{

    Требуется для атаки с помощью техники спекулятивного выполнения:

    \begin{enumerate}
    \item<1-|alert@1-2> Примитив спекулятивного выполнения
    \item<4-|alert@4-5> Гаджеты для создания «окна» спекулятивного выполнения
    \item<7-|alert@7-8> Гаджеты обнародования информации
    \item<10-|alert@10-> Примитив обнародования информации
    \end{enumerate}
  }

  \note<1>{

    А так ли всё легко и просто? Представим, что мы хотим совершить атаку.

    Рассмотрим на примере Spectre v2 (variant 2).

    Процессор \textbf{должен поддерживать} возможность спекулятивного
    выполнения.

    Требуется \textbf{найти место} в коде, где может происходить спекулятивное
    выполнение по желанию атакующего.

  }

  \note<4>{

    Требуется найти гаджеты, которые позволят создать достаточно
    \textbf{длительное по времени выполнения окно} для спекулятивного
    выполнения.
    
  }

  \note<7>{

    Требуется найти гаджеты, которые позволят \textbf{считать необходимую
      закрытую информацию} в ходе спекулятивного выполнения.
    
  }

  \note<10>{

    Требуется возможность \textbf{считать полученную} через сторонний канал
    информацию или \textbf{удостовериться}, что она была считана.

  }

  \only<3, 6, 9, 12>{
    \begin{figure}[h]
      \begin{tikzpicture}[
        align = center,
        ->,
        > = Stealth,
        thick,
        block/.style = {
          draw,
          fill = ForestGreen,
          text = White,
        },
        ]

        \node[
        block,
        ] (type_bp) {
          Вид ПП
        };

        \node[
        block,
        right = of type_bp,
        rotate = 3,
        ] (algo_bp) {
          Алгоритм работы ПП
        };

        \node[
        block,
        right = of algo_bp,
        rotate = -2,
        ] (work_bp) {
          Характерные условия работы ПП
        };

        \node<6->[
        block,
        above = 0 of $(type_bp.north)!0.5!(algo_bp.north)$,
        rotate = 2,
        ] (search_cache_gadgets) {
          Поиск/создание гаджетов
        };

        \node<6->[
        block,
        above = .13 of $(work_bp.north)!0.5!(algo_bp.north)$,
        % rotate = 1,
        ] (info_cache) {
          Содержимое кэша
        };

        \node<9->[
        block,
        above = .05 of $(search_cache_gadgets.north)!0.5!(info_cache.north)$,
        rotate = 4,
        ] (bypass_aslr) {
          Обход ASLR
        };

        \node<9->[
        block,
        above = 0 of info_cache.north east,
        rotate = -8,
        ] (bypass_others) {
          Обход других систем защиты
        };

        \node<12->[
        block,
        above = 2.2 of search_cache_gadgets,
        rotate = 80,
        ] (timers) {
          Высокоточные таймеры
        };

        \node<12->[
        block,
        above = -0.5 of bypass_aslr,
        minimum width = 5cm,
        minimum height = 2cm,
        fill opacity = 0.5,
        text opacity = 1,
        text = Black,
        ] (noise) {
          Шум
        };

        \node<12->[
        block,
        above = 0 of noise,
        ] (flush_reload) {
          Flush + Reload
        };

        \node<12->[
        inner sep = 0,
        above = 1 of bypass_others.east,
        ] (vovka) {
          \includegraphics[width = .15\textwidth]{vovka}
        };

      \end{tikzpicture}
      \only<1-5>{\caption{Фундамент башни\\\textit{атаки на основе
            спекулятивного выполнения}}}

      \only<6-8>{\caption{Башня\\\textit{атаки на основе спекулятивного
            выполнения}}}

      \only<9->{\caption{\textbf{Вавилонская} башня\\\textit{атаки на основе
            спекулятивного выполнения}}}
    \end{figure}
  }

  \note<3>{

    Выберем тренировку предсказателя переходов. Мы столкнёмся:

    \begin{itemize}
    \item требуется знать \textbf{вид предсказателя переходов}
    \item требуется знать \textbf{алгоритм работы предсказателя переходов}
    \item для \textbf{разных процессоров --- разные условия}, например, в i7 два
      буфера предсказателя переходов.
    \end{itemize}

    В whitepaper и в PoC даны \textbf{примеры для конкретных процессоров}.

  }

  \note<6>{

    Выберем гаджеты для загрузки некэшированных данных. Мы столкнёмся:

    \begin{itemize}
    \item в случае JIT --- создание гаджетов, в других случаях гаджеты следует
      искать,
    \item что хранится в кэше на данный момент.
    \end{itemize}

    В whitepaper и в PoC даны \textbf{примеры для конкретных процессоров}.

  }

  \note<9>{

    В whitepaper и в PoC \textbf{все защиты отключены}.

  }

  \note<12>{

    Что? Ещё одна атака?

  }

  \only<2>{

    \begin{itemize}
    \item Обход проверки границ
    \item Тренировка предсказателя переходов
    \item Чтение памяти после сохранения её в регистр
    \item Отложенная исключительная ситуация
    \item Засорение таблиц с историей шаблонов переходов \insm
    \item Засорение Return Stack Buffer
    \item Спекулятивная запись (buffer overflow)
    \end{itemize}
    
  }

  \note<2> {

    На данный момент существует несколько техник, позволяющих производить
    спекулятивное выполнение по желанию атакующего.

    Для того, чтобы использовать те или иные техники \textbf{требуется
      досконально знать микроархитектуру процессора}.
    
  }

  \only<5>{

    \begin{itemize}
    \item Загрузка некэшированных данных
    \item Цепочка из зависимых загрузок данных
    \item Цепочка из зависимых целочисленных операций в АЛУ
    \end{itemize}
    
  }

  \note<5>{

    Не \textbf{везде есть такие цепочки} в окружении. В некоторых случаях
    \textbf{требуется знать, какие данные сейчас в кэше}.
    
  }

  \only<8>{

    \begin{itemize}
    \item ASLR
    \item CFI
    \item SMAP
    \item DEP/NX
    \item retpoline
    \item И т. д.
    \end{itemize}
    
  }

  \note<8>{

    Для применения необходимых гаджетов требуется обойти некоторые системы
    защиты.
    
  }

  \only<11>{

    \begin{itemize}
    \item Устройство кэша
    \item Правила вымещения из кэша
    \item Эксклюзивность и инклюзивность
    \item Тип атаки
    \item Зашумлённость
    \item Счётчики
    \item И т. д.
    \end{itemize}

  }

  \note<11>{

    \begin{itemize}
    \item Требуется знать, какой тип кэша используется.
    \item Требуется знать, какие данные сейчас хранятся в кэше, как выталкивать.
    \item На какой кэш будет направлена атака.
    \item Возможность проведения той или иной атаки.
    \item Возможность многократного повторения атаки.
    \item Для измерения времени требуются высокоточные счётчики.
    \end{itemize}

  }

\end{frame}
