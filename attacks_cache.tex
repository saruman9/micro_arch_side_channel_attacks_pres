\begin{frame}{\insertsubsection}

  \begin{figure}[h]
    \begin{tikzpicture}[
      align = center,
      ->,
      > = Stealth,
      thick,
      block/.style = {
        draw,
        fill = ForestGreen,
        text = White,
      },
      ]

      \node[
      block,
      label = above:Кэш,
      minimum width = 1cm,
      minimum height = 1cm,
      ] (cache) {
        Данные
      };

      \node[
      block,
      label = above:DRAM,
      below = of cache,
      minimum width = 3cm,
      minimum height = 3cm,
      ] (dram) {
        Данные
      };

      \node[
      block,
      right = 3cm of $(cache.east)!.5!(dram.east)$,
      minimum width = 2cm,
      minimum height = 2cm,
      ] (cpu) {
        CPU
      };

      \draw (cpu) -> node[above, sloped] {0.5 ns -- 7 ns} (cache);
      \draw (cpu) -> node[below, sloped] {250 ns} (dram);

    \end{tikzpicture}
    \caption{Кэш --- это не только полезно, но и опасно}\label{cache_faster}
  \end{figure}

  \note{

    Главное \textbf{предназначение кэш--памяти} --- \textbf{нивелировать
      задержки} при работе с медленной главной физической памятью. При работе с
    данными через кэш задержки значительно уменьшаются. Kocher в 1996 году
    описал возможность использования разницы во времени при обращении к данным
    для совершения атаки, которая стала называться \textbf{атака на кэш по
      времени}.

  }
\end{frame}


% \begin{frame}{\insertsubsection}

%   \begin{figure}[h]
%     \begin{tikzpicture}[
%       align = center,
%       ->,
%       > = Stealth,
%       thick,
%       block/.style = {
%         draw,
%         fill = ForestGreen,
%         text = White,
%       },
%       ]

%       \node[
%       block,
%       label = above:Кэш,
%       minimum width = 1cm,
%       minimum height = 1cm,
%       ] (cache) {
%         Данные
%       };

%       \node[
%       block,
%       right = 3cm of cache,
%       minimum width = 2cm,
%       minimum height = 2cm,
%       ] (cpu) {
%         CPU
%       };

%       \draw (cpu) -> node[above] {
%         \includegraphics[height = .1\textheight]{hourglass} = ?
%       } (cache);

%     \end{tikzpicture}
%     \caption{Для атаки на кэш необходимо знать точное время цикла обращения к
%       ячейке памяти}\label{cache_time_attack}
%   \end{figure}

%   \note{

%     Для проведения успешной кеш-атаки необходимо \textbf{знать точное время
%       цикла обращения к ячейке памяти}. Ранние кеш--атаки использовали для этих
%     целей \textbf{системные счётчики производительности}, но этот способ
%     \textbf{неэффективен}, поскольку эти счётчики на ARM-процессорах доступны
%     только в привилегированном режиме.

%     Однако в 2016 году были предложены \textbf{три альтернативных источника}
%     синхронизации, доступные в том числе и в непривилегированном режиме. Один из
%     них --- запуск \textbf{параллельного синхронизирующего потока}, который
%     непрерывно инкрементирует глобальную переменную. Читая значение этой
%     переменной, злоумышленник может измерять время цикла обращения к ячейке
%     памяти. Об остальных методах \textbf{будет рассказано позже} в ходе
%     повествования.

%     Кроме того, в \textbf{ARM--процессорах} действует так называемая политика
%     псевдослучайного замещения, в результате действия которой,
%     \textbf{вытеснение из кеша происходит менее предсказуемо}, чем в процессорах
%     Intel и AMD. Тем не менее в 2016 году была \textbf{продемонстрирована
%       эффективная кеш-атака} даже в таких зашумленных условиях.

%   }
% \end{frame}

% \subsubsection{Evict + Time}
% \begin{frame}{\insertsubsubsection}

%   \begin{enumerate}
%   \item Измерить время выполнения программы--жертвы
%   \item Вытеснить определённый набор кэша
%   \item Снова измерить время выполнения программы--жертвы и сравнить
%   \end{enumerate}

%   \note{

%     Атакующий запускает несколько вычислений и затем измеряет время,
%     потребовавшееся на завершение данных вычислений. Для определения влияния на
%     конкретный набор кэша, атакующий предварительно вытесняет этот набор перед
%     вторым запуском вычислений. Для второй половины запусков вычислений
%     атакующий производит сравнение времени, и в случае, если время вычислений
%     изменилось, значит именно этот набор кэша использовался для вычислений.

%   }

% \end{frame}

% \begin{frame}{\insertsubsubsection}

%   \only<1>{Измерить время выполнения программы--жертвы}

%   \only<2,4>{Вытеснить определённый набор кэша}

%   \only<3,5>{Снова измерить время выполнения программы--жертвы и сравнить}

%   \begin{figure}[h]

%   \newcommand*\viccolor{ForestGreen}
%   \newcommand*\attcolor{White}
%   \newcommand*\attcolorone{White}

%   \only<4,5>{
%     \renewcommand*\viccolor{BrickRed}
%   }
%   \only<2,3>{
%     \renewcommand*\attcolor{NavyBlue}
%   }

%   \only<4,5>{
%     \renewcommand*\attcolorone{NavyBlue}
%   }

%     \begin{tikzpicture}[
%       align = center,
%       ->,
%       > = Stealth,
%       thick,
%       ampersand replacement = \&,
%       block/.style = {
%         draw,
%         fill = ForestGreen,
%         text = White,
%       },
%       cache/.style = {
%         matrix of nodes,
%         nodes in empty cells,
%         nodes = {
%           minimum width = .5cm,
%           minimum height = .5cm,
%           draw,
%           align = center,
%           anchor = center,
%         },
%       },
%       ]

%       \node<1,3,5>[
%       font = \ttfamily,
%       text = Black,
%       align = left,
%       ] (victim) {
%         long long *rsa\_encrypt(...)
%         \{\\
%           $\cdots$\\
%           for (i = 0; i < size; i++)\{\\
%             encrypted[i] = rsa\_modExp(\\
%               message[i],\\
%               pub->exponent,\\
%               pub->modules);\\
%           \}\\
%           $\cdots$\\
%         \}\\
%       };

%       \node<1,3,5>[
%       above = .1 of victim,
%       ] (time) {
%         \only<1>{0.30 ms}
%         \only<3>{0.28 ms}
%         \only<5>{0.56 ms}
%       };

%       \node[
%       cache,
%       right = of victim,
%       label = {Кэш (8 наборов, 4 пути)},
%       ] (cache) {
%         \node[fill = ForestGreen] {}; \& \& \& \\
%          \& \& \& \\
%          \& \& \node[fill = ForestGreen] {}; \& \\
%          \& \& \& \\
%          \& \node[fill = ForestGreen] {}; \& \& \node[fill = ForestGreen] {}; \\
%          \& \& \& \\
%         \node[fill = \attcolorone] (attack_1) {}; \& \node[fill = \attcolorone] {}; \& \node[fill = \viccolor] (detect) {}; \& \node[fill = \attcolorone] {}; \\
%         \node[fill = \attcolor] (attack_0) {}; \& \node[fill = \attcolor] {}; \& \node[fill = \attcolor] {}; \& \node[fill = \attcolor] {}; \\
%       };

%       \node<2,4,5>[
%       left = of cache,
%       ] (bsd) {
%         \includegraphics[height = .1\textheight]{freebsd}
%       };

%       \draw<2>[red] (bsd) -> (attack_0.west);
%       \draw<4>[red] (bsd) -> (attack_1.west);
%       \draw<5>[red] (bsd) -> (detect.center);

%     \end{tikzpicture}
%     % \caption{Пример атаки Evict + Time}\label{evict_time_example}
%   \end{figure}

%   \note<1>{

%     \textbf{Кэш должен быть прогретым.}

%   }

%   \note<5>{

%     \texttt{Evict + Time} метод даёт детальную информацию об используемом наборе
%     кэша, но \textbf{«шум»}, получаемый от других источников, мешает корректно
%     вычислить время исполнения задачи. По этой причине может потребоваться
%     \textbf{несколько тысяч повторений запусков} для получения, например, целого
%     AES ключа. Данный метод требует от атакующего возможности \textbf{точного
%       измерения времени начала и окончания вычислений}. Это может оказаться
%     \textbf{неосуществимым} в случае выполнения атаки \textbf{асинхронно}, так
%     как атакующий не имеет возможности запускать вычисления по своему
%     усмотрению. Преимущество \texttt{Evict + Time} метода в том, что ему
%     \textbf{не требуется работать с разделяемой памятью}.

%     \textbf{Усложнённые адресация данных и правила вымещения из кэша} на
%     современных процессорах делает этап очищения данных из набора кэша сложнее
%     и, соответственно, саму атаку с помощью метода \texttt{Evict + Time}.

%   }
% \end{frame}

% \subsubsection{Prime + Probe}
% \begin{frame}{\insertsubsubsection}
%   \begin{enumerate}
%     \item Заполнить определённые наборы кэша
%     \item Передать управление программе--жертве
%     \item Определить какие наборы кэша всё ещё заполнены нашими данными
%   \end{enumerate}

%   \note{

%     Вторая техника атаки представленная Osvik'ом оказалось более мощной.
%     Атакующий в течении времени заполняет (primes) набор кэша, а затем измеряет
%     как долго происходит повторный доступ к этим данным.

%     Первое применение атаки было \textbf{нацелено на L1 кэш}. Однако,
%     \textbf{обратная разработка} работы \textbf{кэшей последнего уровня} открыла
%     возможность проведения атаки и на данный вид кэшей (2009). Существует
%     множество примеров успешной атаки на криптографические алгоритмы из состава
%     всё того же OpenSSL.

%     2011 --- был представлен пример атаки на \textbf{соседние облака и
%       прослушивание соседних виртуальных машин}. Однако, эти атаки были
%     совершены на микроархитектуры \textbf{с простой организацией}, где
%     отсутствовали срезы кэшей и сложные функции по вычислению адресов.

%     Позже --- и на современные процессоры, в том числе атака на AES BouncyCastle
%     (\textbf{ARM}), атака на \textbf{TrustZone}, \textbf{Intel SGX анклав}.

%   }
% \end{frame}

% \begin{frame}{\insertsubsubsection}

%   \only<1>{Заполнить определённые наборы кэша}

%   \only<2>{Передать управление программе--жертве}

%   \only<3-6>{Определить какие наборы кэша всё ещё заполнены нашими данными}

%   \begin{figure}[h]

%     \newcommand*\viccolor{White}
%     \newcommand*\vicatcolorone{White}
%     \newcommand*\vicatcolortwo{White}
%     \newcommand*\attcolor{NavyBlue}
%     \newcommand*\attcolorone{White}

%     \only<1>{
%       \renewcommand*\vicatcolorone{NavyBlue}
%       \renewcommand*\vicatcolortwo{NavyBlue}
%     }
%     \only<2-7>{
%       \renewcommand*\viccolor{ForestGreen}
%     }
%     \only<2-4>{
%       \renewcommand*\vicatcolorone{ForestGreen}
%     }
%     \only<2-6>{
%       \renewcommand*\vicatcolortwo{ForestGreen}
%     }
%     \only<4->{
%       \renewcommand*\vicatcolorone{BrickRed}
%     }
%     \only<6->{
%       \renewcommand*\vicatcolortwo{BrickRed}
%     }

%     \begin{tikzpicture}[
%       align = center,
%       ->,
%       > = Stealth,
%       thick,
%       ampersand replacement = \&,
%       block/.style = {
%         draw,
%         fill = ForestGreen,
%         text = White,
%       },
%       cache/.style = {
%         matrix of nodes,
%         nodes in empty cells,
%         nodes = {
%           minimum width = .5cm,
%           minimum height = .5cm,
%           draw,
%           align = center,
%           anchor = center,
%         },
%       },
%       caption/.style = {
%         fill = White,
%         sloped,
%         above,
%         inner sep = 0,
%         outer sep = 3,
%       },
%       ]

%       \node[
%       cache,
%       label = {Кэш (8 наборов, 4 пути)},
%       ] (cache) {
%         \&\&\&\\
%         \& \node[fill = \viccolor] (vic_0) {}; \&\&\\
%         \&\&\&\\
%         \&\&\&\\
%         \&\&\&\\
%         \& \node[fill = \viccolor] (vic_1) {}; \& \node[fill = \viccolor] (vic_2) {}; \&\\
%         \&\&\&\\
%         \node[fill = \attcolor] (att_0) {}; \& \node[fill = \vicatcolorone] (vic_3) {}; \& \node[fill = \attcolor] (att_1) {}; \& \node[fill = \vicatcolortwo] (vic_4) {}; \\
%       };

%       \node[
%       left = of cache,
%       ] (bsd) {
%         \includegraphics[height = .1\textheight]{freebsd}
%       };

%       \node[
%       right = of cache,
%       ] (smile) {
%         \includegraphics[height = .1\textheight]{smile}
%       };

%       \node<7>[
%       right = of cache,
%       yshift = 5,
%       ] (aim) {
%         \includegraphics[height = .1\textheight]{aim}
%       };

%       \draw<1>[red] (bsd) -> (att_0.west);
%       \draw<2>[LimeGreen] (smile) -> (vic_0.center);
%       \draw<2>[LimeGreen] (smile) -> (vic_1.center);
%       \draw<2>[LimeGreen] (smile) -> (vic_2.center);
%       \draw<2>[LimeGreen] (smile) -> (vic_3.center);
%       \draw<2>[LimeGreen] (smile) -> (vic_4.center);
%       \draw<3>[red] (bsd) -> node[caption] {hit} (att_0.center);
%       \draw<4>[red] (bsd) -> node[caption] {miss} (vic_3.center);
%       \draw<5>[red] (bsd) -> node[caption] {hit} (att_1.center);
%       \draw<6>[red] (bsd) -> node[caption] {miss} (vic_4.center);
%       \draw<7>[red] (bsd) -> node[caption] {Detected!} (att_0.west);

%     \end{tikzpicture}
%     % \caption{Пример атаки Prime + Probe}\label{prime_probe_example}
%   \end{figure}

%   \note<1> {

%     Атака проиллюстрирована в трёх шагах. Атакующий непрерывно заполняет набор
%     кэша, используя доступ к своей памяти и позднее измеряет время доступа (шаг
%     1 и 3).

%   }

%   \note<2>{

%     На шаге 2 жертва, возможно, обращается к участку памяти (не общей), которая
%     расположена в том же наборе кэша.

%   }

%   \note<3-6>{

%     Если жертва обращалась в тот же набор кэша, то время доступа к данным на
%     шаге 3 увеличится, так как один из путей кэша будет заменён, в противном
%     случае время доступа будет меньше.

%     \textbf{Время}, затрачиваемое на повторный доступ к набору кэша
%     \textbf{пропорционально количеству путей}, которые были заменены другими
%     процессами. \textbf{Большое время} при измерении означает, что \textbf{по
%       крайней мере один путь кэша был заменён}, и наоборот --- меньшее время
%     указывает на то, что ни один из путей кэша не был заменён.

%   }

%   \note<7>{

%     Атака имеет такую же \textbf{детализацию} направленности атаки, как и Evict
%     + Time атака, т. е. \textbf{набор кэша}. \textbf{Точность атаки больше}, чем
%     в случае с Evict + Time, так как измеряется время \textbf{прямого доступа к
%       данным}, а Evict + Time измеряет время через запуски вычислений, а не
%     напрямую. Но также, как и с Evict + Time атакой, \textbf{усложнение функций
%       адресации и правил вымещения} из кэша делает проведение атаки
%     затруднительным.

%   }

% \end{frame}

\subsubsection{Flush + Reload}
\begin{frame}{\insertsubsubsection}
  \begin{enumerate}
  \item Отобразить бинарный файл (например, разделяемый объект) в своё адресное
    пространство
  \item Сбросить содержимое кэш--линии (код или данные)
  \item Передать управление программе--жертве
  \item Определить какие линии кэша были загружены программой--жертвой снова
  \end{enumerate}

  \note{

    Данная атака считается \textbf{наиболее эффективной} атакой на кэш. Целью
    данной атаки является не просто набор кэша, а \textbf{отдельная линия кэша},
    более того, у атакующего существует возможность проверить закэширована ли та
    или иная область памяти. Атака \texttt{Flush + Reload} выполняется в три
    фазы.

  }
\end{frame}

\begin{frame}{\insertsubsubsection}

  \only<1>{Отобразить бинарный файл (например, разделяемый объект) в своё
    адресное пространство}%

  \only<2>{Сбросить содержимое кэш--линии (код или данные)}%

  \only<3>{Передать управление программе--жертве}%

  \only<4>{Определить какие линии кэша были загружены программой--жертвой
    снова}%

  \begin{figure}[h]

    \newcommand*\viccolor{ForestGreen}
    \newcommand*\vicatcolorone{ForestGreen}
    \newcommand*\vicatcolortwo{ForestGreen}
    \newcommand*\attcolor{NavyBlue}
    \newcommand*\attcolorone{White}

    \only<2>{
      \renewcommand*\vicatcolorone{White}
      \renewcommand*\vicatcolortwo{White}
    }
    \only<3>{
      \renewcommand*\vicatcolorone{White}
      \renewcommand*\vicatcolortwo{ForestGreen}
    }
    \only<4->{
      \renewcommand*\vicatcolorone{NavyBlue}
      \renewcommand*\vicatcolortwo{BrickRed}
    }

    \begin{tikzpicture}[
      align = center,
      ->,
      > = Stealth,
      thick,
      ampersand replacement = \&,
      block/.style = {
        draw,
        fill = ForestGreen,
        text = White,
      },
      cache/.style = {
        matrix of nodes,
        nodes in empty cells,
        nodes = {
          minimum width = .5cm,
          minimum height = .5cm,
          draw,
          align = center,
          anchor = center,
        },
      },
      caption/.style = {
        fill = White,
        sloped,
        above,
        inner sep = 0,
        outer sep = 3,
      },
      ]

      \node[
      cache,
      label = {Кэш (8 наборов, 4 пути)},
      ] (cache) {
        \& \& \& \\
        \& \node[fill = \vicatcolorone] (vic_0) {}; \& \& \\
        \& \& \& \\
        \& \& \node[fill = \vicatcolortwo] (vic_1) {}; \& \\
        \& \& \& \\
        \& \& \& \\
        \& \& \& \\
        \& \& \& \\
      };

      \node[
      left = of cache,
      ] (bsd) {
        \includegraphics[height = .1\textheight]{freebsd}
      };

      \node[
      right = of cache,
      ] (smile) {
        \includegraphics[height = .1\textheight]{smile}
      };

      \node<5>[
      right = of cache,
      yshift = 5,
      ] (aim) {
        \includegraphics[height = .1\textheight]{aim}
      };

      \draw<1>[Green] (smile) -> (vic_0);
      \draw<1>[Green] (smile) -> (vic_1);
      \draw<1>[red] (bsd) -> (vic_0);
      \draw<1>[red] (bsd) -> (vic_1);
      \draw<2>[red] (bsd) -> node[caption] {flush} (vic_0);
      \draw<2>[red] (bsd) -> node[caption] {flush} (vic_1);
      \draw<3>[Green] (smile) -> (vic_1);
      \draw<4>[red] (bsd) -> node[caption] {miss} (vic_0);
      \draw<4>[red] (bsd) -> node[caption] {hit} (vic_1);
      \draw<5>[red] (bsd) -> node[caption] {Detected!} (vic_1);

    \end{tikzpicture}
    % \caption{Пример атаки Flush + Reload}\label{flush_reload_example}
  \end{figure}

  \note<1>{

    \texttt{Flush + Reload} атака работает при условии \textbf{общей памяти},
    ярким примером может служить общая библиотека, которую использует и
    атакующий и программа--жертва. По этой причине, в случае, если нет такого
    элемента, как общая память между атакующим и жертвой, придётся использовать
    схему атаки \texttt{Prime + Probe}.

  }

  \note<2>{

    Во время первой фазы контролируемый участок памяти удаляется из структуры
    кэша (удаляется линия кэша с помощью инструкции \textbf{clflush} в случае с
    Intel).

  }


  \note<3>{

    Во второй фазе программа--шпион находится в режиме ожидания, \textbf{давая
      жертве время воспользоваться} участком памяти.

  }

  \note<4>{

    Во время третьей фазы программа--шпион \textbf{перезагружает} участок памяти
    и \textbf{замеряет время} загрузки. Если во время второй фазы жертва
    \textbf{воспользовалась} участком памяти, то этот участок будет доступен из
    кэша и операция перезагрузки \textbf{пройдёт быстро}. С другой стороны, если
    линия кэша \textbf{осталась неиспользованной}, понадобится время на загрузку
    участка и операция перезагрузки пройдёт \textbf{значительно дольше}.

  }


  \note<5>{

    Такие методы атаки, как \texttt{Flush + Reload} и \texttt{Flush + Flush}
    (описан ниже), используют \textbf{непривилегированную} x86--инструкцию
    сброса \texttt{clflush} для удаления строки данных из кеш--памяти. Однако,
    за исключением процессоров ARMv8-A, \textbf{ARM--платформы не имеют
      непривилегированных инструкций сброса кеша}, и поэтому в 2016 году был
    предложен \textbf{косвенный метод вытеснения кеша}, с использованием эффекта
    Rowhammer.

  }
\end{frame}

\subsubsection{Другие типы атак на кэш}
\begin{frame}{\insertsubsubsection}
  \begin{itemize}
  \item Evict + Time
  \item Prime + Probe
  \item Prime + Abort
  \item Flush + Flush
  \item Evict + Reload
  \item AnC (ASLR $\oplus$ Cache)
  \item и др.
  \end{itemize}

  \note{

    Существуют другие атаки на кэш, которые применяются в различных ситуациях,
    позволяют проводить атаку в стелс-режиме, направлены на различные уровни
    кэшей и т. д.
    
  }
\end{frame}

% \subsubsection{Flush + Flush}
% \begin{frame}{\insertsubsubsection}

%   \begin{figure}[h]
%     \begin{tikzpicture}[
%       align = center,
%       ->,
%       > = Stealth,
%       thick,
%       ampersand replacement = \&,
%       block/.style = {
%         draw,
%         fill = ForestGreen,
%         text = White,
%       },
%       cache/.style = {
%         matrix of nodes,
%         nodes in empty cells,
%         nodes = {
%           minimum width = .5cm,
%           minimum height = .5cm,
%           draw,
%           align = center,
%           anchor = center,
%         },
%       },
%       caption/.style = {
%         fill = White,
%         sloped,
%         above,
%         inner sep = 0,
%         outer sep = 3,
%       },
%       ]

%       \node[
%       cache,
%       label = {Кэш (8 наборов, 4 пути)},
%       ] (cache) {
%         \& \& \& \\
%         \& \node[fill = ForestGreen] (vic_0) {}; \& \& \\
%         \& \& \& \\
%         \& \& \node[fill = ForestGreen] (vic_1) {}; \& \\
%         \& \& \& \\
%         \& \& \& \\
%         \& \& \& \\
%         \& \& \& \\
%       };

%       \node[
%       left = of cache,
%       ] (bsd) {
%         \includegraphics[height = .1\textheight]{freebsd}
%       };

%       \node[
%       right = of cache,
%       text = Black,
%       label = {[White] below:Я слежу за тобой!},
%       ] (smile) {
%         \includegraphics[height = .1\textheight]{smile}
%       };

%       \draw[red] (bsd) -> (vic_0);
%       \draw[red] (bsd) -> (vic_1);
%       \draw[red] (bsd) -> (cache-1-1.center);
%       \draw[red] (bsd) -> (cache-1-2.center);
%       \draw[red] (bsd) -> (cache-3-1.center);
%       \draw[red] (bsd) -> (cache-5-3.center);
%       \draw[red] (bsd) -> (cache-6-1.center);
%       \draw[red] (bsd) -> (cache-7-2.center);

%     \end{tikzpicture}
%     \caption{Количество и продолжительность обращений к памяти может быть
%       измерено, а атаки на кэш --- обнаружены}\label{cache_attacks_detection}
%   \end{figure}

%   \note{

%     Атаки типа \texttt{Flush + Reload} и \texttt{Prime + Probe} производят
%     \textbf{большое количество обращений} к памяти, продолжительность которых
%     \textbf{можно измерить} (при помощи системных счётчиков производительности),
%     по этой причине они \textbf{могут быть опознаны} процессором.

%     Атака \textbf{Flush + Flush} представляет из себя \textbf{стелс--версию}
%     атаки \texttt{Flush + Reload}.

%   }

% \end{frame}

% \begin{frame}{\insertsubsubsection}

%   \begin{figure}[h]

%     \begin{tikzpicture}[
%       align = center,
%       ->,
%       > = Stealth,
%       thick,
%       ampersand replacement = \&,
%       block/.style = {
%         draw,
%         fill = ForestGreen,
%         text = White,
%       },
%       cache/.style = {
%         matrix of nodes,
%         nodes in empty cells,
%         nodes = {
%           minimum width = .5cm,
%           minimum height = .5cm,
%           draw,
%           align = center,
%           anchor = center,
%         },
%       },
%       caption/.style = {
%         fill = White,
%         sloped,
%         above,
%         inner sep = 0,
%         outer sep = 3,
%       },
%       ]

%       \node[
%       cache,
%       label = {Кэш (8 наборов, 4 пути)},
%       ] (cache) {
%         \& \& \& \\
%         \& \node[fill = ForestGreen] (vic_0) {}; \& \& \\
%         \& \& \& \\
%         \& \& \node[fill = ForestGreen] (vic_1) {}; \& \\
%         \& \& \& \\
%         \& \& \& \\
%         \& \& \& \\
%         \& \& \& \\
%       };

%       \node[
%       left = of cache,
%       ] (bsd) {
%         \includegraphics[height = .1\textheight]{freebsd}
%       };

%       \draw[red] (bsd) -> node[caption, text = BlueViolet] {3.83ns} (vic_0);
%       \draw[red] (bsd) -> node[caption, text = BlueViolet] {4.85ns} (vic_1);
%       \draw[red] (bsd) -> node[caption, text = Black] {0.96ns} (cache-7-4.center);
%       \draw[red] (bsd) -> node[caption, text = Black] {1.06ns} (cache-8-1.center);

%     \end{tikzpicture}
%     \caption{Инструкция для сброса кэша срабатывает за различное время в
%       зависимости от того, находятся ли сейчас какие-либо данные в кэше или
%       нет}\label{flush_flush_example}
%   \end{figure}

%   \note{

%     Было выяснено, что \textbf{инструкция для сброса кэша срабатывает за
%       различное время} в зависимости от того, находятся ли сейчас какие-либо
%     данные в кэше или нет. В случае, если данные, которые подвергаются
%     вытеснению \textbf{присутствуют}, то вытеснение происходит
%     \textbf{медленнее}. Таким образом вместо повторной загрузки данных в кэш,
%     атакующий снова вытесняет кэш и также измеряет время.

%   }
% \end{frame}

% \subsubsection{Evict + Reload}
% \begin{frame}{\insertsubsubsection}

%   \only<1>{Отобразить бинарный файл (например, разделяемый объект) в своё
%     адресное пространство}

%   \only<2>{Вытеснить содержимое кэш--линии (код или данные)}

%   \only<3>{Передать управление программе--жертве}

%   \only<4>{Определить какие линии кэша были загружены программой--жертвой снова}

%   \begin{figure}[h]

%     \newcommand*\viccolor{ForestGreen}
%     \newcommand*\vicatcolorone{ForestGreen}
%     \newcommand*\vicatcolortwo{ForestGreen}
%     \newcommand*\attcolor{White}
%     \newcommand*\attcolorone{White}

%     \only<2>{
%       \renewcommand*\vicatcolorone{NavyBlue}
%     }
%     \only<2->{
%       \renewcommand*\attcolor{NavyBlue}
%       \renewcommand*\vicatcolortwo{NavyBlue}
%     }
%     \only<3>{
%       \renewcommand*\vicatcolorone{BrickRed}
%       % \renewcommand*\vicatcolortwo{ForestGreen}
%     }
%     \only<4->{
%       \renewcommand*\vicatcolorone{BrickRed}
%       % \renewcommand*\vicatcolortwo{ForestGreen}
%     }

%     \begin{tikzpicture}[
%       align = center,
%       ->,
%       > = Stealth,
%       thick,
%       ampersand replacement = \&,
%       block/.style = {
%         draw,
%         fill = ForestGreen,
%         text = White,
%       },
%       cache/.style = {
%         matrix of nodes,
%         nodes in empty cells,
%         nodes = {
%           minimum width = .5cm,
%           minimum height = .5cm,
%           draw,
%           align = center,
%           anchor = center,
%         },
%       },
%       caption/.style = {
%         fill = White,
%         sloped,
%         above,
%         inner sep = 0,
%         outer sep = 3,
%       },
%       ]

%       \node[
%       cache,
%       label = {Кэш (8 наборов, 4 пути)},
%       ] (cache) {
%         \& \& \& \\
%         \node[fill = \attcolor] (evict_0) {}; \& \node[fill = \vicatcolorone] (vic_0) {}; \& \node[fill = \attcolor] {}; \& \node[fill = \attcolor] {}; \\
%         \& \& \& \\
%         \node[fill = \attcolor] (evict_1) {}; \& \node[fill = \attcolor] {}; \& \node[fill = \vicatcolortwo] (vic_1) {}; \& \node[fill = \attcolor] {}; \\
%         \& \& \& \\
%         \& \& \& \\
%         \& \& \& \\
%         \& \& \& \\
%       };

%       \node[
%       left = of cache,
%       ] (bsd) {
%         \includegraphics[height = .1\textheight]{freebsd}
%       };

%       \node[
%       right = of cache,
%       ] (smile) {
%         \includegraphics[height = .1\textheight]{smile}
%       };

%       \node<5>[
%       right = of cache,
%       yshift = 5,
%       ] (aim) {
%         \includegraphics[height = .1\textheight]{aim}
%       };

%       \draw<1>[green] (smile) -> (vic_0);
%       \draw<1>[green] (smile) -> (vic_1);
%       \draw<1>[red] (bsd) -> (vic_0);
%       \draw<1>[red] (bsd) -> (vic_1);
%       \draw<2>[red] (bsd) -> node[caption] {evict} (evict_0.west);
%       \draw<2>[red] (bsd) -> node[caption] {evict} (evict_1.west);
%       \draw<3>[green] (smile) -> (vic_0);
%       \draw<4>[red] (bsd) -> node[caption] {hit} (vic_0);
%       \draw<4>[red] (bsd) -> node[caption] {miss} (vic_1);
%       \draw<5>[red] (bsd) -> node[caption] {Detected!} (vic_0);

%     \end{tikzpicture}
%     % \caption{Пример атаки Flush + Reload}\label{flush_reload_example}
%   \end{figure}

%   \note<1>{

%     Атака представляет из себя \textbf{модифицированную версию Flush + Reload} и
%     имеет смысл для архитектур, где \textbf{инструкция вытеснения} из кэша
%     доступна \textbf{только в привилегированном режиме}, например, для ARM.

%   }

%   \note<2>{

%     Суть атаки заключается в том, что \textbf{для вытеснения} нужной линии кэша
%     происходит \textbf{заполнение кэш--памяти большим количеством
%       взаимосвязанных адресов}, в результате чего механизм, отвечающий за
%     вытеснение, сам начнёт вытеснять нужную нам линию кэша.

%   }

%   \note<5>{

%     Новый (2017 год) тип атаки на кэш от
%     \href{https://www.vusec.net/projects/anc/}{Systems and Network Security
%       Group at VU Amsterdam}.

%     Атака представляет из себя \textbf{модифицированную версию Evict + Time}.
%     Были использованы \textbf{новые методы для расчёт времени} (Time to Tick ---
%     ввиду ограничений на стандартный счётчик, который теперь отсчитывает такты,
%     не измеряя точное время доступа к памяти, отсчитываются такты после
%     обращения к памяти; Shared Memory Counter --- параллельный счётчик на
%     отдельном ядре; пример на рисунке \ref{fig:arc_new_timers}). Также
%     \textbf{использовалась атака Prime + Probe} на MMU (Memory Management Unit),
%     для возможности проведения атаки на кэш последнего уровня.

%   }

% \end{frame}