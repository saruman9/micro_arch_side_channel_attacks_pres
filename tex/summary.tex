\begin{frame}{\insertsection}

  \begin{itemize}[<+->]
  \item Software-based microarchitectural attacks has become \color{dBlue}{very popular}
  \item Requires {\color{dBlue}a lot of resources} to develop a working exploit
  \item Microarchitectural attacks may be {\color{dBlue}automated}
  \item Many attacks have {\color{dBlue}not yet been disclosed}
  \item Countermeasures come with a {\color{dBlue}performance impact}
  \end{itemize}

  \note<1>{

    Всё больше исследований проводится в этой области, всё больше обычных
    обывателей интересуются данной проблемой. Уязвимости в ПО \textbf{всё
      сложнее эксплуатировать, переходим к железу}.

    Уязвимости находят, \textbf{прочитав и разобравшись в спецификации
      архитектуры}, процессора. Обратную разработку производят с помощью базовых
    атак по сторонним каналам.
    
  }

  \note<2>{

    Не смотря на все многочисленные плюсы (для атакующего), \textbf{разработка
      эксплоита} для широкого спектра программного и аппаратного ПО ---
    \textbf{весьма затруднительна}.
    
  }

  \note<3>{

    Представлено множество работ и инструментов, позволяющих провести атаку
    практически на любую популярную архитектуру. \textbf{Создаются
      эксплоит-паки}, содержащие атаки на микроархитектуру.
    
  }

  \note<4>{

    Описание атак, использующих спекулятивное выполнение, ещё \textbf{не до
      конца опубликованы}.

    Множество возможных \textbf{изъянов микроархитектуры не найдены}.
    
  }

  \note<5>{

    Для исправления сложившейся ситуации требуются \textbf{фундаментальные
      изменения} в ходе работы процессора.

    Исправления, \textbf{разработанные на уровне ОС}, требуют \textbf{детального
      изучения уязвимости} и алгоритма противодействия, к тому же \textbf{не
      всегда возможно предотвратить} эксплуатацию на уровне ОС, а если и
    удаётся, то в большинстве случаев приносят \textbf{в жертву процессы
      оптимизации}.
    
  }

\end{frame}

\begin{frame}[standout]
  \color{dWhite}Questions?
\end{frame}