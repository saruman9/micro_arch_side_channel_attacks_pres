\begin{frame}{\insertsubsection}

  CVE-2018-3639: спекулятивное выполнение чтения памяти после сохранения её в
  регистр

  \begin{figure}[h]
    \includegraphics[height = .7\textheight]{spectre_logo}
    % \caption{Spectre}
  \end{figure}

  \note{


  }
\end{frame}

% \subsubsection{Сначала чтение, потом запись}
% \begin{frame}[fragile]{\insertsubsubsection}
%   \begin{minted}{nasm}
%     STR X1, [X2]   ; X2 - адрес памяти, который ещё не известен
%     ...
%     LDR X3, [X4]   ; X4 содержит тот же адрес, что и X2
%     <произвольная обработка X3>
%     LDR X5, [X6, X3]  ; спекулятивное выполнение со старым адресом
%   \end{minted}

%   \note{

%     Во многих современных процессорах применяются интересные техники
%     оптимизации, а именно: \textbf{загрузка данных} из памяти по определённому
%     адресу производится \textbf{раньше, чем запись} в тот же участок памяти в
%     случае, \textbf{если адрес} на этапе записи данных \textbf{ещё не известен}.

%     В итоге у нас \textbf{спекулятивно выполняются} все операции \textbf{со
%       старым адресом} в регистре (в том числе запись в кэш).

%   }
% \end{frame}

% \subsubsection{Читаем данные EL1}
% \begin{frame}[fragile]{\insertsubsubsection}
%   \begin{minted}{nasm}
%     STR X1, [X2]
%     ...
%     ERET           ; возврат на более нижний уровень исключений
%     ...
%     LDR X3, [X4]   ; X4 содержит такой же физический адрес, как и X2,
%                    ; но виртуальный адрес отличается
%     <произвольная обработка X3>
%     LDR X5, [X6, X3]
%   \end{minted}

%   \note{

%     Если \textbf{адрес один и тот же} (виртуальный и физический), то данная
%     уязвимость эксплуатируема только \textbf{на одном уровне исключений}
%     (exception level for ARM).

%     В случае, если есть возможность \textbf{чтения на одном уровне исключений},
%     а \textbf{загрузки на другом}, и при этом используется \textbf{один и тот же
%       физический адрес}, то есть возможность эксплуатации уязвимости \textbf{на
%       разных уровнях}.

%   }
% \end{frame}

% \subsubsection{Спекулятивное чтение одного и того же регистра}
% \begin{frame}[fragile]{\insertsubsubsection}
%   \begin{minted}{nasm}
%     STR X1, [SP]
%     ...
%     LDR X3, [SP]
%     <произвольная обработка X3>
%     LDR X5, [X6, X3]
%     <произвольная обработка X5>
%     LDR X7, [X8, X5]
%   \end{minted}

%   \note{

%     Удивительно, но факт --- мы можем читать данные из памяти, обращаясь по
%     одному и тому же регистру.

%     Такое возможно благодаря \textbf{современным оптимизациям на некоторых
%       архитектурах}. В случае, если у нас содержимое SP, например,
%     \textbf{отсутствовало в кэше}, то \textbf{при записи} мы получим
%     \textbf{cache miss} и задержку, позволяющую нам \textbf{спекулятивно
%       прочитать всё те же данные}. Такое возможно по причине того, что процессор
%     в RoB отслеживает, когда данные из регистра попали в кэш и соответственно
%     ускоряет исполнение инструкций.

%     \textbf{PoC нет!}

%   }
% \end{frame}

% \subsubsection{Спекулятивный запуск непривилегированного кода}
% \begin{frame}[fragile]{\insertsubsubsection}
%   \begin{minted}{nasm}
%     STR X1, [SP]
%     ...
%     LDR X3, [SP]
%     ...
%     BLR X3
%   \end{minted}

%   \note{

%     Так же, как и в обычном Spectre, мы можем составить цепочку ROP гаджетов и
%     спекулятивно их запустить, чтобы \textbf{записать нужные нам данные в кэш}.

%   }
% \end{frame}

% \subsubsection{Сначала запись, потом чтение}
% \begin{frame}[fragile]{\insertsubsubsection}
%   \begin{minted}{nasm}
%     ...
%     LDR X3, [X4]
%     <произвольная обработка X3>
%     LDR X5, [X6, X3]
%     ....
%     STR X1, [X2] ; X2 содержит тот же адрес, что и X4
%   \end{minted}

%   \note{

%     Утверждается, что существует возможность прочитать данные спекулятивно,
%     записанные также спекулятивно, но позже. Это возможно в случае, если регистр
%     для чтения будет высчитываться гораздо дольше регистра для записи.

%     \textbf{PoC нет!}

%   }
% \end{frame}

\subsubsection{Предотвращение}
\begin{frame}{\insertsubsubsection}

  Частично такое же, как и в случае с Variant 2

  \begin{itemize}
  \item отключение реорганизации операций чтения и записи
  \item SafeSpec
  \end{itemize}

  \note{

    Медленно!

  }

\end{frame}
