% \begin{frame}{\insertsubsection}

%   \begin{figure}[h]

%     \newcommand*\dramcolor{White, White, Green, NavyBlue, Green, White, White}

%     \newcommand*\bufcolor{White}

%     \only<2-7>{
%       \renewcommand*\bufcolor{Green}
%     }

%     \begin{tikzpicture}[
%       align = center,
%       ->,
%       > = Stealth,
%       thick,
%       dram/.style = {
%         rectangle,
%         rectangle split,
%         rectangle split parts = 7,
%         rectangle split part fill = \dramcolor,
%         minimum width = 3cm,
%         draw,
%         text = Black,
%         text centered,
%       },
%       block/.style = {
%         rectangle,
%         draw,
%         fill = ForestGreen,
%         text = White,
%         text centered,
%       },
%       ]

%       \node[
%       dram,
%       label = above:DRAM банк,
%       ] (dram) {
%         01011101
%         \nodepart{two}
%         10101011
%         \nodepart{three}
%         00010010
%         \nodepart{four}
%         \only<1-7>{10111101}%
%         \only<8>{\color{red}0\color{Black}011\color{red}0\color{Black}1\color{red}1\color{Black}1}%
%         \nodepart{five}
%         00111111
%         \nodepart{six}
%         10000000
%         \nodepart{seven}
%         00000001
%         \nodepart{eight}
%       };

%       \node[
%       below = .5 of dram,
%       text = Black,
%       fill = \bufcolor,
%       draw,
%       minimum width = 3cm,
%       ] (buf) {
%         \only<1>{row buffer}%
%         \only<2>{00111111}%
%         \only<3>{00010010}%
%         \only<4>{00111111}%
%         \only<5>{00010010}%
%         \only<6>{00111111}%
%         \only<7>{00010010}%
%         \only<8>{row buffer}%
%       };

%       \node[
%       right = of dram.five east,
%       ] (attack_0) {
%         \only<1>{Память атакующего}%
%         \only<2,4,6>{Очистка кэша\\Чтение}%
%       };

%       \node[
%       right = of dram.three east,
%       ] (attack_1) {
%         \only<1>{Память атакующего}%
%         \only<3,5,7>{Очистка кэша\\Чтение}%
%       };

%       \node[
%       left = of dram,
%       ] (attack_2) {
%         \only<1>{Память жертвы}%
%       };

%       \draw<1> (attack_0) -- (dram.five east);
%       \draw<1> (attack_1) -- (dram.three east);
%       \draw<1> (attack_2) -- (dram.four west);

%       \draw<2>[Green] (attack_0) -- (dram.five east);
%       \draw<2> (dram.five west) -- node[above left] {Открытие\\строки} ++(-1, 0) |- (buf.west);

%       \draw<3>[Green] (attack_1) -- (dram.three east);
%       \draw<3> (dram.three west) -- node[above left] {Открытие\\строки} ++(-1, 0) |- (buf.west);

%       \draw<4>[Green] (attack_0) -- (dram.five east);
%       \draw<4> (dram.five west) -- node[above left] {Открытие\\строки} ++(-1, 0) |- (buf.west);

%       \draw<5>[Green] (attack_1) -- (dram.three east);
%       \draw<5> (dram.three west) -- node[above left] {Открытие\\строки} ++(-1, 0) |- (buf.west);

%       \draw<6>[Green] (attack_0) -- (dram.five east);
%       \draw<6> (dram.five west) -- node[above left] {Открытие\\строки} ++(-1, 0) |- (buf.west);

%       \draw<7>[Green] (attack_1) -- (dram.three east);
%       \draw<7> (dram.three west) -- node[above left] {Открытие\\строки} ++(-1, 0) |- (buf.west);

%     \end{tikzpicture}
%     \only<1>{\caption{Для эксплуатации атака должна быть направлена на память,
%         расположенную в одном и том же банке, но в разных строках}}
%     \only<2-7>{\caption{«Долбим» память вокруг памяти жертвы}}
%     \only<8>{\caption{В результате --- самопроизвольное переключение битов
%         памяти}}
%   \end{figure}

%   \note<1>{

%     Первая атака, основанная на дефекте микроархитектуры, названном впоследствии
%     \textbf{Rowhammer}, была найдена Kim в 2014 году. Она запускалась из
%     программного обеспечения и могла наносить вред безопасности системы.

%   }

%   \note<2>{

%     Атакующий постоянно \textbf{получает доступ к DRAM памяти}, при этом
%     \textbf{сбрасывает кэш} для того, чтобы \textbf{часто открывать-закрывать
%       строки} DRAM. Если строки DRAM находятся в физической близости друг с
%     другом, то на одной из строк может \textbf{самопроизвольно переключиться
%       бит}. Память, в которой самопроизвольно переключается бит, могла быть
%     недоступна атакующему и даже принадлежать другому домену безопасности.

%   }
% \end{frame}

% \begin{frame}{\insertsubsection}

%   \begin{figure}[h]
%     \begin{tikzpicture}[
%       align = center,
%       ->,
%       > = Stealth,
%       thick,
%       ampersand replacement = \&,
%       dram/.style = {
%         matrix of nodes,
%         nodes in empty cells,
%         text = Black,
%         nodes = {
%           minimum width = .5cm,
%           minimum height = .5cm,
%           draw,
%           align = center,
%           anchor = center,
%           outer sep = 0,
%         },
%       },
%       block/.style = {
%         rectangle,
%         draw,
%         fill = ForestGreen,
%         text = White,
%         text centered,
%       },
%       ]

%       \only<2>{\tikzset{row 3/.style={red}}}
%       \only<3>{\tikzset{
%           row 3 column 1/.style = {red},
%           row 3 column 3/.style = {red},
%           row 3 column 5/.style = {red},
%           row 2/.style = {red},
%           row 4/.style = {red},
%         }}

%       \node[
%       dram,
%       label = above:DRAM,
%       ] (dram) {
%        1\&1\&1\&0\&1\&1\&0\&0\\
%        1\&0\&0\&1\&0\&0\&0\&0\\
%        \only<1-2>{0}\only<3>{1}\&% change
%        \only<1,3>{1}\only<2>{0}\&
%        \only<1>{1}\only<2-3>{0}\&% change
%        0\&
%        \only<1-2>{0}\only<3>{1}\&% change
%        0\&
%        \only<1,3>{1}\only<2>{0}\&
%        0\\
%        1\&0\&0\&0\&1\&1\&0\&0\\
%        0\&0\&1\&0\&0\&0\&0\&1\\
%       };

%       \node<1>[
%       right = of dram.north east,
%       ] (cap) {
%         Конденсатор
%       };

%       \node<2>[
%       right = of dram-3-8,
%       ] (open_3) {
%         Открытие строки
%       };

%       \node<3>[
%       right = of dram-4-8,
%       ] (open_4) {
%         Открытие строки
%       };

%       \node<3>[
%       right = of dram-2-8,
%       ] (open_2) {
%         Открытие строки
%       };

%       \draw<1> (cap) -- (dram-1-8);
%       \draw<2> (open_3) -- (dram-3-8);
%       \draw<3> (open_2) -- (dram-2-8);
%       \draw<3> (open_4) -- (dram-4-8);

%     \end{tikzpicture}
%     \only<1>{\caption{Что происходит при Rowhammer?}}
%     \only<2>{\caption{На всю строку подаётся питание}}
%     \only<3>{\caption{Активация строк в современных DRAM}}
%   \end{figure}

%   \note<1>{

%     Почему же так происходит? Каждая ячейка DRAM --- это конденсатор; 0 и 1 ---
%     это \textbf{заряженное или разряженное} состояние конденсатора. Каждая
%     ячейка в сетке \textbf{связана с соседней ячейкой проводом}.

%   }

%   \note<2>{

%     Если какая-либо ячейка активируется, то напряжение подаётся как на её
%     конденсатор, так и на \textbf{все остальные конденсаторы той же строки}.

%   }

%   \note<3>{

%     Поскольку ячейки памяти по мере технологического прогресса становятся всё
%     меньше и меньше и \textbf{всё ближе} друг к другу, помехи, вызванные
%     активацией строки памяти, очень часто \textbf{влияют на заряды конденсаторов
%       соседних} строк.

%     Первые реализации атак с использованием эффекта Rowhammer полагались либо на
%     вероятностные методы (из-за чего в процессе атаки могло произойти
%     незапланированное обрушение системы), либо на специализированные функции
%     управления памятью: дедупликацию памяти, паравиртуализацию MMU, интерфейс
%     pagemap. Однако подобные функции на современных устройствах либо недоступны
%     вообще, либо отключены по соображениям безопасности.

%   }

% \end{frame}

\subsubsection{Rowhammer. Exploitation primitives}
\begin{frame}{\insertsubsubsection}
  \begin{itemize}
  \item Fast uncached memory access
  \item Physical memory massaging
  \item Physical memory addressing
  \end{itemize}

  \note{

    \textbf{Быстрый некэшируемый доступ к памяти.} Данный примитив не так-то
    легко получить, т. к. \textbf{контроллер памяти} на CPU может
    \textbf{недостаточно быстро обрабатывать запросы} на чтение памяти.
    Нерасторопность контроллера, как правило нивелируется загрузкой данных в
    кэши. \textbf{Использование кэшей также необходимо предотвратить} для того,
    чтобы было постоянное обращение к DRAM.

    \textbf{Определение местонахождения уязвимых строк.} Кроме того, требуется,
    чтобы жертва использовала нужные атакующему строки DRAM для хранения важной
    информации.

    \textbf{Знание функций адресации физической памяти.} Требуется для
    определения методов трансляции виртуальных адресов в физические и трансляции
    физических адресов на аппаратное средство.

  }
\end{frame}

\subsubsection{Variations of Rowhammer}
\begin{frame}{\insertsubsubsection}

  \begin{itemize}
  \item Flip Feng Shui --- targeted Rowhammer
  \item Throwhammer --- remote Rowhammer
  \item Nethammer --- better remote Rowhammer
  \item Drammer, RAMpage --- exploitation ARM-based hardware
  \item Glitch --- better exploitation ARM-based hardware
  \end{itemize}

  \note{

    Очень много НО.

    \begin{itemize}
    \item дедупликация памяти, \textbf{разделяемая память}.
    \item \textbf{удалённый прямой доступ к памяти (remote direct memory access,
        RDMA)}
    \item \textbf{Intel CAT}, \textbf{некэшируемая память}, инструкции очистки
      кэша в сетевых драйверах.
    \item ARM --- медленная запись в память, инструкция очистки кэша ---
      привилегированная, \textbf{Android ION аллокатор памяти}.
    \item \textbf{GPU на мобильных устройствах}, т. к. CPU и GPU используют одну
      оперативную память, используется WebGL.
    \end{itemize}

  }

  % \only<1>{
  %   \begin{itemize}
  %   \item[$+$] методика «массажирования памяти» для атаки на конкретный адрес
  %   \item[$-$] при условии наличия систем разделения памяти (дедупликация,
  %     виртуальные машины и т. п.)
  %   \end{itemize}
  % }

  % \only<2>{
  %   \begin{itemize}
  %   \item[$+$] удалённо
  %   \item[$-$] remote direct memory access (RDMA)
  %   \end{itemize}
  % }

  % \only<3>{
  %   \begin{itemize}
  %   \item[$+$] удалённо
  %   \item[$-$] Intel CAT
  %   \item[$-$] драйверы сетевых устройств используют инструкции очистки кэша
  %   \item[$-$] используется некэшируемая память
  %   \end{itemize}
  % }


  % \only<4>{
  %   \begin{itemize}
  %   \item[$-$] ARMv7 --- непривилегированный сброс кэша невозможен
  %   \item[$-$] ARMv8 --- инструкция сброса кэша отключена на уровне ядра
  %   \item[$-$] системный вызов \texttt{cacheflush()} --- сброс кэша только до
  %     второго уровня
  %   \item[$-$] вытеснение из кэша с помощью вычислений --- медленно
  %   \item[$+$] Android ION allocator --- некэшируемая DMA память
  %   \end{itemize}
  % }

  % \only<5>{
  %   \begin{itemize}
  %   \item[\checkmark] механизм вычисления времени доступа к памяти и другим ресурсам --- WebGL
  %   \item[\checkmark] общие ресурсы --- кэш GPU
  %   \item[\checkmark] знание физического расположения данных в памяти GPU --- обратная
  %     разработка с помощью атак по сторонним каналам (примитивы представлены
  %     выше)
  %   \item[\checkmark] быстрый доступ к памяти --- WebGL + GPU
  %   \end{itemize}
  % }

  % \note<1>{

  %   В 2016 году была представлена методика \textbf{«массажа памяти» Flip Feng
  %     Shui (FFS)} --- новый вектор эксплуатации эффекта Rowhammer, который
  %   позволяет злоумышленнику \textbf{возбуждать предсказуемые битовые перескоки}
  %   в произвольном месте физической памяти и иметь полный контроль над этим
  %   процессом даже при полном отсутствии уязвимостей в атакуемом программном
  %   обеспечении. В рамках демонстрации методики FFS \textbf{скомпрометирован
  %     механизм обновления}, используемый операционными системами
  %   \textbf{Ubuntu/Debian}. Компрометация удаётся в случае, \textbf{если
  %     используется дедупликация памяти, виртуальные машины работают на одной
  %     системе или используется разделяемая память}.

  % }

  % \note<2>{

  %   В 2018 была описана новая атака --- \textbf{Throwhammer}: Rowhammer Attacks
  %   over the Network and Defenses. Она позволяла проводить атаки типа Rowhammer
  %   \textbf{удалённо}, достаточно было послать необходимые сетевые пакеты.
  %   Единственное условие --- машины должны быть подключены \textbf{через
  %     удалённый прямой доступ к памяти (remote direct memory access, RDMA)}, что
  %   характерно только для высокопроизводительных кластеров и облачных
  %   технологий.

  % }

  % \note<3>{

  %   В 2018 году была описана ещё одна удалённая атака использующая Rowhammer ---
  %   \textbf{Nethammer}: Inducing Rowhammer Faults through Network Requests.
  %   Данная атака позволяла изменять биты памяти с помощью \textbf{специально
  %     сконфигурированного сетевого пакета}, который вызывал чтение по
  %   установленным участкам памяти.

  %   Для Rowhammer атаки требуется также очистка кэша. В данной работе этот
  %   вопрос «решён» тремя способами:

  %   \begin{enumerate}
  %   \item на жертве используется \textbf{Intel CAT технология}, которая
  %     размещает данные в определённом порядке, что позволяет атакующему
  %     вытеснять из кэша необходимые ему данные;
  %   \item драйверы сетевых устройств используют \textbf{инструкции очистки кэша}
  %     (\texttt{clflush}, например);
  %   \item на жертве \textbf{используется некэшируемая память}, поэтому очищать
  %     кэш не требуется вовсе.
  %   \end{enumerate}

  % }

  % \note<4>{

  %   \textbf{Drammer} --- новый вид атаки, способный проводит атаку Rowhammer на
  %   мобильных устройствах. До этой атаки утверждалось, что ARM устройства якобы
  %   неуязвимы по причине \textbf{медленной скорости чтения памяти}.

  %   Как уже известно, для атаки подобной Rowhammer \textbf{требуется проводить
  %     сброс кэша}. На устройствах на \textbf{ARMv7} чипе произвести сброс кэша
  %   \textbf{невозможно}. Однако, в ядре Android существует системный вызов
  %   \textbf{cacheflush()}, который можно вызвать из пользовательского контекста
  %   выполнения, очищает только до второго уровня --- \textbf{недостаточно} для
  %   атаки на DRAM. Что касается \textbf{ARMv8}, то там присутствуют инструкция,
  %   позволяющая сбрасывать кэш, но она \textbf{отключена на уровне ядра}.

  %   Также существует метод вытеснения из кэша с помощью различных
  %   \textbf{вычислений}, но к сожалению данная атака не подходит по причине
  %   того, что на практике (на архитектурах ARMv7 и ARMv8) вычисления
  %   производятся \textbf{с недостаточной скоростью}.

  %   В итоге был использован \textbf{Android ION аллокатор памяти}, который
  %   позволяет работать с \textbf{некэшируемой DMA памятью}.

  % }

  % \note<5>{

  %   Новый вид атаки типа Rowhammer, направлен на эксплуатацию \textbf{GPU на
  %     мобильных устройствах}. Для атаки \textbf{используется WebGL механизм},
  %   встроенный в браузер (Firefox).

  %   \begin{itemize}
  %   \item \textbf{механизм вычисления времени доступа к памяти} и другим
  %     ресурсам (используется механизм \textbf{WebGL для создания высокоточных
  %       таймеров})
  %   \item \textbf{общие ресурсы} (\textbf{кэш GPU}, работа которого не
  %     документирована, но в результате обратной разработки было выяснено, что
  %     для оптимизации кэша используются весьма простые алгоритмы)
  %   \item знание \textbf{физического расположения данных} (было получено в
  %     результате проведения атак по сторонним каналам)
  %   \item \textbf{быстрый доступ к памяти} (получается за счёт использования
  %     GPU + WebGL)
  %   \end{itemize}

  %   В итоге был разработан эксплоит на JS, позволяющий обойти ASLR механизм,
  %   скомпрометировать данные.

  % }
\end{frame}