\begin{frame}{\insertsubsection}

  \begin{figure}[h]
    \begin{tikzpicture}[
      align = center,
      ->,
      > = Stealth,
      ampersand replacement = \&,
      block/.style = {
        draw = dGreen,
      },
      cache/.style = {
        matrix of nodes,
        nodes in empty cells,
        inner sep = 0,
        nodes = {
          minimum width = 3cm,
          minimum height = .5cm,
          draw = dGray,
          align = center,
          anchor = center,
        },
      },
      ]

      \node[
      cache,
      label = {Branch Target Buffer},
      ] (btb) {
        Address tag \& Target \\%
        \& \\%
        \& \\%
        0xebe45a82 \& ??? \\%
        \& \\%
        \& \\%
        \& \\%
      };

      \node[
      draw,
      left = 7cm of btb.north,
      label = above:Virtual address (user space),
      ] (addr_user) {0x0000 \color{dGreen}EBE45A82};

      \node[
      draw,
      below = 3cm of addr_user,
      label = below:Virtual address (kernel space),
      ] (addr_kernel) {0xFFFF \color{dGreen}EBE45A82};

      \node[
      left = of btb,
      block,
      minimum width = 1cm,
      minimum height = 1cm,
      label = left:Indexing\\function,
      ] (f) {f(x)};

      \draw (addr_user.east) -| (f.north);
      \draw (addr_kernel.east) -| (f.south);
      \draw (f) -> (btb.west);

    \end{tikzpicture}
    \caption{Branch Target Buffer addressing a scheme in the Haswell
      processor}\label{btb_tag}
  \end{figure}

  \note{

    \textbf{Буфер адресов перехода (branch target buffer, BTB)} кэширует
    информацию о ранее выбранных переходах выполнения для быстрого угадывания
    будущих. Кэш использует индексацию на \textbf{базе виртуального адресного
      пространства}, таким образом \textbf{атакующему не обязательно знать
      физический адрес} для выполнения атаки.

    \textbf{Атакующий заполняет буфер адресов перехода} путём выполнения
    последовательности различных переходов. Если жертва--программа будет
    выполнять ту ветвь, которой не было в кэше, то она добавится туда, вытеснив
    тем самым существующую запись. \textbf{Атакующий может вычислить} какая
    ветвь была вытолкнута из буфера по сравнительно \textbf{большому времени
      выполнения} этой ветви.

    Пример атаки позволяет взломать KASLR из пользовательского процесса.
    Основывается на возникающих \textbf{коллизиях} в кэше branch target buffer.
    По времени выполнения своего кода атакующий имеет возможность
    \textbf{вычислить адрес перехода в ядерном пространстве} (значение берётся
    из BTB), тем самым вычислить смешение, полученное в результате KASLR.

    Возникает два типа коллизий:

    \begin{enumerate}
    \item cross domain collisions --- user и kernel space;
    \item same domain collisions --- разные user-space процессы, позволяет
      выявить смещения для ASLR другого процесса.
    \end{enumerate}

  }

\end{frame}