\begin{frame}{\insertsubsection}

  \begin{itemize}
  \item Scheduler interrupts
  \item Instruction aborts
  \item Page faults
  \item Behavioral differences (e.g, error code)
  \end{itemize}

  \note{

    Данного рода атаки получают необходимую информацию из исключительных
    ситуаций, которые происходят при работе процессора. Обычными для процессора
    исключительными ситуациями являются: прерывание планировщика, прерывания
    инструкции, ошибка страницы памяти, а также поведенческие изменения,
    например, инструкции предоставляют пользователю код ошибки.

    Во время возникновения исключительных ситуаций можно получить информацию о
    работе процессора либо \textbf{напрямую} (основываясь на поведении самого
    процессора), либо \textbf{косвенно (через измерения времени, при
      возникновении исключительных ситуаций)}.

    Одним из ярких примеров атак подобного рода является \textbf{атака на
      систему дедупликации памяти}.

  }
\end{frame}

% \subsubsection{Атаки на систему дедупликации памяти}
% \begin{frame}{\insertsubsubsection}

%   \begin{figure}[h]

%     \begin{tikzpicture}[
%       align = center,
%       ->,
%       > = Stealth,
%       thick,
%       ampersand replacement = \&,
%       block/.style = {
%         draw,
%         fill = ForestGreen,
%         text = White,
%       },
%       block_split/.style = {
%         rectangle,
%         rectangle split,
%         rectangle split parts = 7,
%         rectangle split empty part width = 1cm,
%         draw,
%         text centered,
%         font = \scriptsize,
%       },
%       ]

%       \node[
%       block_split,
%       label = above:Виртуальная\\память 1,
%       rectangle split part fill = {
%         White, White, White, ForestGreen, White,
%         ForestGreen, White
%       },
%       ] (virt_0) {
%         \nodepart{two}
%         \nodepart{three}
%         \nodepart{four}
%         \nodepart{five}
%         \nodepart{six}
%         \nodepart{seven}
%         \nodepart{eight}
%       };

%       \node<3-4>[
%       block_split,
%       label = above:Физическая\\память,
%       right = 2cm of virt_0,
%       rectangle split part fill = {
%         White, White, White, White, ForestGreen,
%         White, ForestGreen
%       },
%       ] (phys) {
%         \nodepart{two}
%         \nodepart{three}
%         \nodepart{four}
%         \nodepart{five}
%         \nodepart{six}
%         \nodepart{seven}
%         \nodepart{eight}
%       };


%       \node<3-4>[
%       block_split,
%       label = above:Разделяемая\\физическая память,
%       right = 2cm of phys,
%       rectangle split part fill = {
%         White, ForestGreen, White, White, Blue,
%         White, White
%       },
%       ] (dedup) {
%         \nodepart{two}
%         \nodepart{three}
%         \nodepart{four}
%         \nodepart{five}
%         \nodepart{six}
%         \nodepart{seven}
%         \nodepart{eight}
%       };

%       \node<3-4>[
%       block_split,
%       label = above:Виртуальная\\память 2,
%       right = 2cm of dedup,
%       rectangle split part fill = {
%         ForestGreen, White, White, ForestGreen, White,
%         White, White
%       },
%       ] (virt_1) {
%         \nodepart{two}
%         \nodepart{three}
%         \nodepart{four}
%         \nodepart{five}
%         \nodepart{six}
%         \nodepart{seven}
%         \nodepart{eight}
%       };

%       \node<1-2>[
%       block_split,
%       label = above:Разделяемая\\физическая память,
%       right = 3cm of virt_0,
%       rectangle split part fill = {
%         White, ForestGreen, White, White, Blue,
%         White, ForestGreen
%       },
%       ] (phys) {
%         \nodepart{two}
%         \nodepart{three}
%         \nodepart{four}
%         \nodepart{five}
%         \nodepart{six}
%         \nodepart{seven}
%         \nodepart{eight}
%       };

%       \node<1-2>[
%       block_split,
%       label = above:Виртуальная\\память 2,
%       right = 3cm of phys,
%       rectangle split part fill = {
%         ForestGreen, White, White, ForestGreen, White,
%         White, White
%       },
%       ] (virt_1) {
%         \nodepart{two}
%         \nodepart{three}
%         \nodepart{four}
%         \nodepart{five}
%         \nodepart{six}
%         \nodepart{seven}
%         \nodepart{eight}
%       };

%       \draw<1-2> (virt_0.four east) -> (phys.five west);
%       \draw<1-2> (virt_0.six east) -> (phys.seven west);
%       \draw<1-2> (virt_1.four west) -> (phys.five east);
%       \draw<1-2> (virt_1.one west) -> (phys.two east);

%       \draw<2> (virt_0.four east) -> node[above, sloped] {Page fault!} node[below, sloped] {Запись} (phys.five west);

%       \draw<3> (virt_0.four east) -> (phys.five west);
%       \draw<3> (virt_0.six east) -> (phys.seven west);
%       \draw<3> (virt_1.four west) -> (dedup.five east);
%       \draw<3> (virt_1.one west) -> (dedup.two east);

%       \draw<4> (virt_1.four west) -> (dedup.five east);
%       \draw<4> (virt_1.one west) -> (dedup.two east);
%       \draw<4> (virt_0.four east) -> node[below, sloped] {Запись} node[above, sloped] {244ns} (phys.five west);
%       \draw<4> (virt_0.four west) -- ++(-1, 0) -- ++(0, -2) -- node[below] {Запись} node[above] {275ns} ++(7, 0) |- (dedup.five west);


%     \end{tikzpicture}
%     \only<1>{\caption{Дедупликация памяти}}\label{dedup_example}

%     \only<2>{\caption{Процесс 1 пытается записать данные в разделяемую память}}

%     \only<3>{\caption{Запись в дедуплицированную память происходит в режиме
%         Copy-on-Write}}
%   \end{figure}

%   \note<1>{

%     \textbf{Подсистема дедупликации} контент--ориентированных страниц
%     \textbf{сканирует всю системную память на идентичные физические} страницы
%     памяти и \textbf{склеивает их в одну} физическую страницу.

%   }

%   \note<2>{

%     Подсистема дедупликации проецирует несколько \textbf{идентичных копий
%       физических страниц} памяти \textbf{на одну} разделяемую копию, с доступом
%     в режиме \textbf{«копирование при записи»}. В результате при запросах на
%     чтение каждый процесс получает данные из одной и той же страницы. Если же
%     процесс хочет записать данные, то перед тем, как он сможет это сделать, для
%     него создаётся отдельная копия страницы. В результате \textbf{запись в
%       разделённую страницу вызывает «страничный отказ»}.

%   }

%   \note<4> {

%     Следовательно, \textbf{запись в разделяемую страницу происходит значительно
%       медленнее}, чем запись в обычную страницу. Злоумышленник,
%     \textbf{способный создавать страницы в целевой системе}, может использовать
%     эту разницу во времени, чтобы обнаружить факт \textbf{существования
%       интересующих его страниц}.

%   }

% \end{frame}