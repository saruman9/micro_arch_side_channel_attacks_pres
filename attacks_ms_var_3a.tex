\begin{frame}{\insertsubsection}

  CVE-2018-3060: спекулятивное чтение недоступных данных

  \begin{figure}[h]
    \includegraphics[height = .7\textheight]{meltdown_logo}
    \caption{Meltdown}
  \end{figure}

  \note{

    По сути всё тот же meltdown.


  }
\end{frame}

\subsubsection{ASM}
\begin{frame}[fragile]{\insertsubsubsection}

  \begin{minted}{nasm}
    LDR X1, [X2]       ; X2 - указатель на данные, которых нет в кэше,
                       ; также в TLB не должно быть данного адреса
    CBZ X1, over       ; переход, который в итоге будет совершён,
                       ; но инструкции ниже всё равно исполнятся
    MRS X3, TTBR0_EL1  ; TTBR0_EL1 - системный регистр,
                       ; недоступный для чтения пользователем
    LSL X3, X3, #imm   ; получение нужного бита данных
    AND X3, X3, #0xFC0 ; выравнивание с размером страницы памяти
    LDR X5, [X6,X3]    ; X6 - адрес массива атакующего
    over
  \end{minted}

  \note{

    Всё также, как и в случае с чтением недоступной пользователю памяти, только
    здесь производится \textbf{спекулятивное чтение системного регистра}.

    К \textbf{ARM процессорам не применимо}, так как даже в ходе спекулятивного
    выполнения данные из системных регистров не читаются.

  }

\end{frame}

\subsubsection{Предотвращение}
\begin{frame}{\insertsubsubsection}

  В зависимости от уровня привилегий заменять значения системных регистров
  фиктивными

  \note{

    + предотвращение Variant 3.

  }

\end{frame}