\begin{frame}{\insertsubsection}

  CVE-2017-5753: обход проверки границ

  \begin{figure}[h]
    \includegraphics[height = .7\textheight]{spectre_logo}
    \caption{Spectre}
  \end{figure}

  \note{


  }
\end{frame}

\subsubsection{Спекулятивное выполнение}
\begin{frame}{\insertsubsubsection}

  \begin{itemize}
  \item CPU пытается предугадать будущие переходы
    \begin{itemize}
    \item ... учась на произошедших
    \end{itemize}
  \item происходит спекулятивное выполнение инструкций выбранного перехода
  \item если переход угадан верно
    \begin{itemize}
    \item ... быстрое выполнение
    \end{itemize}
  \item если переход угадан неверно
    \begin{itemize}
    \item ... отброс результата спекулятивного выполнения
    \end{itemize}
  \end{itemize}

  \note{

    Вскользь уже была упомянута тема с предугадыванием переходов.

    Какие же \textbf{побочные эффекты} могут нас ожидать при такой архитектуре?

  }
\end{frame}

\subsubsection{Побочные эффекты}
\begin{frame}{\insertsubsubsection}

  \begin{figure}[h]
    \begin{tikzpicture}[
      draw,
      align = center,
      ->,
      > = Stealth,
      thick,
      double distance = 1pt,
      node distance = .6,
      block/.style = {
        rectangle,
        draw,
        fill = ForestGreen,
        text = White,
        minimum width = 2cm,
      },
      ]

      \node[
      text = Black,
      ] (if_bound) {
        if <чтение памяти согласно границе>
      };

      \node<2->[
      below = of if_bound.west,
      ] (true) {
        True
      };

      \node<7->[
      below = of if_bound.east,
      ] (false) {
        False
      };

      \node<3->[
      below left = of true.west,
      block,
      label = left:Предсказания,
      ] (p_t_true) {
        True
      };

      \node<5->[
      below right = of true.east,
      block,
      ] (p_t_false) {
        False
      };

      \node<10->[
      below left = of false.west,
      block,
      ] (p_f_true) {
        True
      };

      \node<8->[
      below right = of false.east,
      block,
      ] (p_f_false) {
        False
      };

      \node<4->[
      below = of p_t_true,
      ] (r_p_t_true) {
        \includegraphics[height = .1\textheight]{rabbit_b}
      };

      \node<6->[
      below = of p_t_false,
      ] (t_p_t_false) {
        \includegraphics[height = .1\textheight]{turtle_b}
      };

      \node<9->[
      below = of p_f_false,
      ] (r_p_f_false) {
        \includegraphics[height = .1\textheight]{rabbit_b}
      };

      \node<11->[
      below = of p_f_true,
      ] (q_p_f_true) {
        \includegraphics[height = .1\textheight]{question_b}
      };

      \draw<2-> (if_bound) -> (true);
      \draw<3-> (true) -> (p_t_true);
      \draw<4-> (p_t_true) -> (r_p_t_true);
      \draw<5-> (true) -> (p_t_false);
      \draw<6-> (p_t_false) -> (t_p_t_false);
      \draw<7-> (if_bound) -> (false);
      \draw<8-> (false) -> (p_f_false);
      \draw<9-> (p_f_false) -> (r_p_f_false);
      \draw<10-> (false) -> (p_f_true);
      \draw<11-> (p_f_true) -> (q_p_f_true);

    \end{tikzpicture}
    % \caption{}\label{fig:exe_in_order}
  \end{figure}

  \note<1>{

    Рассмотрим, в каких случаях происходит спекулятивное выполнение.

  }

  \note<2>{

    Если чтение участка памяти происходит согласно заявленным границам.

  }

  \note<3>{

    Чтение согласно границам, предсказатель также думал, что чтение будет
    происходить согласно границам:

    \begin{itemize}
    \item спекулятивно выполнится код чтения в соответствии с границами памяти
    \end{itemize}

  }

  \note<4>{

    В итоге выполнение кода произойдёт быстро, так как уже заранее был выполнен,
    результаты получены.

  }

  \note<5>{

    Чтение согласно границам, предсказатель же думал, что чтение памяти выйдет
    за границы.

    \begin{itemize}
    \item спекулятивно будет выполняться последующий за чтением памяти код
    \item спекулятивное выполнение отбросится, выполнение кода будет происходить
      заново
    \end{itemize}

  }

  \note<6>{

    В итоге из-за расхождения реальности с предсказанием, скорость выполнения
    кода будет ниже.

  }

  \note<7>{

    Если чтение участка памяти происходит за заявленными границами памяти.

  }

  \note<8>{

    Чтение участка памяти за границами, предсказатель также думал, что чтение
    будет происходить за границами.

    \begin{itemize}
    \item спекулятивно будет выполняться последующий за чтением памяти код
    \end{itemize}

  }

  \note<9>{

    Код исполнится быстро, так как предсказатель угадал и спекулятивно выполнял
    последующий код.

  }

  \note<10>{

    В случае же, если предсказатель ошибся и решил, что вычисления будут
    происходить в границах памяти:

    \begin{itemize}
    \item спекулятивно выполнится код, который на практике выходит за границы
      памяти
    \end{itemize}

  }

  \note<11>{

    \textbf{Спекулятивно выполнится код}, который на практике \textbf{выходит за
      границы памяти}.

  }

\end{frame}

\subsubsection{Тренировка предсказателя переходов}
\begin{frame}[fragile]{\insertsubsubsection}

  \begin{figure}[h]

    \newcommand*\colorcone{Black}
    \newcommand*\colorctwo{Black}
    \newcommand*\colorcthree{Black}
    \newcommand*\colorcfour{Black}
    \newcommand*\colorcfive{Black}
    \newcommand*\colorcsix{Black}
    \newcommand*\colorcseven{Black}
    \newcommand*\digitnum{0}

    \only<2-3> {
      \renewcommand*\colorcone{BrickRed}
    }
    \only<4-5> {
      \renewcommand*\colorctwo{BrickRed}
      \renewcommand*\digitnum{1}
    }
    \only<6-7> {
      \renewcommand*\colorcthree{BrickRed}
      \renewcommand*\digitnum{2}
    }
    \only<8-9> {
      \renewcommand*\colorcfour{BrickRed}
      \renewcommand*\digitnum{3}
    }
    \only<10-11> {
      \renewcommand*\colorcfive{red}
      \renewcommand*\digitnum{4}
    }
    \only<12-13> {
      \renewcommand*\colorcsix{red}
      \renewcommand*\digitnum{5}
    }
    \only<14-15> {
      \renewcommand*\colorcseven{red}
      \renewcommand*\digitnum{6}
    }

    \begin{tikzpicture}[
      draw,
      align = center,
      ->,
      > = Stealth,
      thick,
      double distance = 1pt,
      node distance = .6,
      block/.style = {
        rectangle,
        draw,
        fill = ForestGreen,
        text = White,
        minimum width = 2cm,
      },
      ]

      \node[
      ] (code) {
        \color{Black}index = \color{Mahogany}\digitnum\color{Black};\\
        \color{Black}char* data = "\color{\colorcone}t\color{\colorctwo}e\color{\colorcthree}x\color{\colorcfour}t\color{\colorcfive}K\color{\colorcsix}E\color{\colorcseven}Y\color{Black}";\\
        \color{Black}if (index < \color{Mahogany}4\color{Black})\\
      };

      \node[
      below left = 3 of code,
      text = Black,
      font = \ttfamily,
      minimum width = 3cm,
      ] (cache) {
        arr1[untrusted]
      };

      \node[
      below = of code,
      text = White,
      fill = ForestGreen,
      draw,
      label = below:Предсказание,
      ] (zero) {
        $\omega = $\\
        \only<1-2>{.50 | .50}%
        \only<3-4>{.60 | .40}%
        \only<5-6>{.69 | .31}%
        \only<7-8>{.76 | .24}%
        \only<9-10>{.81 | .19}%
        \only<11-12>{.77 | .23}%
        \only<13-14>{.70 | .30}%
        \only<15>{.64 | .36}%
      };

      \node[
      below right = 3 of code,
      text = Black,
      font = \ttfamily,
      minimum width = 3cm,
      ] (zero) {
        0
      };

      \draw<1-2,11,13,15> (code) -> node[above, sloped] {true} (cache);
      \draw<1,3-10,12,14> (code) -> node[above, sloped] {false} (zero);

      \draw<2>[NavyBlue] (code) -> node[above, sloped] {false} node[below, sloped] {Спекулятивно} (zero);
      \draw<3>[ForestGreen] (code) -> node[above, sloped] {true} node[below, sloped] {Реально} (cache);

      \draw<4>[NavyBlue] (code) -> node[above, sloped] {true} node[below, sloped] {Спекулятивно} (cache);
      \draw<5>[ForestGreen] (code) -> node[above, sloped] {true} node[below, sloped] {Реально} (cache);

      \draw<6>[NavyBlue] (code) -> node[above, sloped] {true} node[below, sloped] {Спекулятивно} (cache);
      \draw<7>[ForestGreen] (code) -> node[above, sloped] {true} node[below, sloped] {Реально} (cache);

      \draw<8>[NavyBlue] (code) -> node[above, sloped] {true} node[below, sloped] {Спекулятивно} (cache);
      \draw<9>[ForestGreen] (code) -> node[above, sloped] {true} node[below, sloped] {Реально} (cache);

      \draw<10>[BrickRed] (code) -> node[above, sloped] {true} node[below, sloped] {Спекулятивно} (cache);
      \draw<11>[ForestGreen] (code) -> node[above, sloped] {false} node[below, sloped] {Реально} (zero);

      \draw<12>[BrickRed] (code) -> node[above, sloped] {true} node[below, sloped] {Спекулятивно} (cache);
      \draw<13>[ForestGreen] (code) -> node[above, sloped] {false} node[below, sloped] {Реально} (zero);

      \draw<14>[BrickRed] (code) -> node[above, sloped] {true} node[below, sloped] {Спекулятивно} (cache);
      \draw<15>[ForestGreen] (code) -> node[above, sloped] {false} node[below, sloped] {Реально} (zero);

    \end{tikzpicture}
    % \caption{Неосознанная тренировка предсказателя переходов}
  \end{figure}

  \note<1>{



  }

\end{frame}

\subsubsection{Обход проверки границ}
\begin{frame}[fragile]{\insertsubsubsection}

  \begin{minted}{c}
    struct array {
      unsigned long length;
      unsigned char data[];
    };
    struct array *arr1 = ...; /* небольшой массив */
    struct array *arr2 = ...; /* массив размером 0x400 */
    unsigned long untrusted_offset_from_user = ...;
    if (untrusted_offset_from_user < arr1->length) {
      unsigned char value = arr1->data[untrusted_offset_from_user];
      unsigned long index2 = ((value & 1) * 0x100) + 0x200;
      if (index2 < arr2->length) {
        unsigned char value2 = arr2->data[index2];
      }
    }
  \end{minted}

  \note{

    Существует множество вариантов эксплуатации данной уязвимости, рассмотрим
    одну из них.

    Разберём всё по порядку.

  }
\end{frame}

\begin{frame}[fragile]{\insertsubsubsection}

  \begin{minted}{c}
    struct array {
      unsigned long length;
      unsigned char data[];
    };
    struct array *arr1 = ...; /* небольшой массив */
    struct array *arr2 = ...; /* массив размером 0x400 */
  \end{minted}

  \note{

    Объявляется два массива.

    Первый --- целевой массив, в котором будет происходить обход границ.

    Второй --- массив для применения атаки на кэш.

  }
\end{frame}

\begin{frame}[fragile]{\insertsubsubsection}

  \begin{minted}{c}
    /* переменная, управляемая атакующим */
    unsigned long untrusted_offset_from_user = ...;

    /* проверка границ */
    if (untrusted_offset_from_user < arr1->length) {

      /* спекулятивное выполнение, получение значения недоступной памяти */
      unsigned char value = arr1->data[untrusted_offset_from_user];

      /* получение значения бита интересующей области памяти */
      unsigned long index2 = ((value & 1) * 0x100) + 0x200;

      /* атака на кэш */
      unsigned char value2 = arr2->data[index2];
    }
  \end{minted}

  \note{

    Проверка границ происходит как в примере представленном ранее.

    Атака на кэш происходит как в примере, рассказанном ранее.

    Код следующий за проверкой границ называется «гаджетом», как и в случае с
    ROP.

    Многое зависит от устройства кэшей различных уровней, TLB, BTB (branch
    target buffers).

  }
\end{frame}

\subsubsection{ASM}
\begin{frame}[fragile]{\insertsubsubsection}

  \begin{minted}{nasm}
    LDR X1, [X2]      ; X2 - указатель на arr1->length
    CMP X0, X1        ; X0 содержит untrusted_offset_from_user
    BGE out_of_range
    LDRB W4, [X5,X0]  ; X5 содержит arr1->data
    AND X4, X4, #1
    LSL X4, X4, #8
    ADD X4, X4, #0x200
    LDRB X7, [X8, X4] ; X8 содержит arr2->data
    out_of_range
  \end{minted}

  \note{

    Упрощённый asm код

    Спекулятивно можно выполнить так же \textbf{ROP гаджеты}, заставить
    выполнять операции, тем самым считывать данные из \textbf{закрытых
      ресурсов}, например, крипто-чип.

  }

\end{frame}


\subsubsection{Предотвращение}
\begin{frame}{\insertsubsubsection}

  \LARGE

  \begin{itemize}
  \item отключение спекулятивного выполнения
  \item ограничение доступа к высокоточным таймерам
  \item привилегированная очистка кэша
  \item полное изолирование важных данных
  \item вставка инструкций для остановки спекулятивного выполнения
  \end{itemize}

  \note{

    \begin{itemize}
    \item Как отключить? Большое проседание в производительности!
    \item сделали свои таймеры
    \item другие методы очистки
    \item spectre работает и на безопасных анклавах
    \item большое проседание по производительности + всё перекомпилировать
    \end{itemize}

  }

\end{frame}